\documentclass[11pt,answers]{exam}

\usepackage{etex}
\usepackage{amssymb,amsmath,multicol} %<-- InWorksheetExam1 i also have fancyhdr,

\usepackage[metapost]{mfpic}
\usepackage[pdftex]{graphicx}
\usepackage{tabu}

\usepackage{pst-plot}
\usepackage{pgfplots}
\pgfplotsset{compat=1.9}

\usepackage{tikz}
\usepackage{tkz-2d}
\usepackage{tkz-base}
\usetikzlibrary{calc}
\usetikzlibrary{arrows}

\usepackage{systeme}

\usepackage[inline]{enumitem}
\usepackage{refcount}%<-- non in WorksheetExam1

\usepackage{pstricks-add,pst-eucl}
\usepackage{systeme}
\usepackage{setspace}
\usepackage{multicol}


\usepackage[inline]{enumitem}   
\makeatletter
% This command ignores the optional argument for itemize and enumerate lists
\newcommand{\inlineitem}[1][]{%
\ifnum\enit@type=\tw@
    {\descriptionlabel{#1}}
  \hspace{\labelsep}%
\else
  \ifnum\enit@type=\z@
       \refstepcounter{\@listctr}\fi
    \quad\@itemlabel\hspace{\labelsep}%
\fi}
\makeatother


\def\f{x+1} \def\g{-x/3+2}  \def\h{-x+3}

\newcommand{\vasymptote}[2][]{
    \draw [densely dashed,#1] ({rel axis cs:0,0} -| {axis cs:#2,0}) -- ({rel axis cs:0,1} -| {axis cs:#2,0});
}

\boxedpoints

\addpoints
%\printanswers
\noprintanswers

\opengraphsfile{Q7b_Fa18}

\begin{document}
\extrawidth{-0.3in}
\pagestyle{headandfoot}

\setlength{\hoffset}{-.25in}

\extraheadheight{-.3in}
\runningheadrule
\firstpageheader{\bfseries {Precalculus}}{ \bfseries {Quiz 7 }}{\bfseries {11/6/18}} 

\begin{center}
	This quiz has \numquestions\ questions, for a total of \numpoints\
	points and \numbonuspoints\ bonus points.
\end{center}


\firstpagefooter{} %%&&CHANGED
                {}
                {%Points earned: \hbox to 0.5in{\hrulefill}
                % out of  \pointsonpage{\thepage} points
                }
                 
						

\vspace*{0.1cm}
\hbox to \textwidth { \scshape {Name:} \enspace\hrulefill}
\vspace{0.1cm}




\pointpoints{point}{points}

\begin{questions}


\addpoints



\question[2] The present value of a sum of money is the amount that must be invested now, at a given rate of interest, to produce the desired sum at a later date. 

Find the present value of \$10,000 if interest is paid at a rate of 0.8\% per year, compounded quarterly, for 3 years. Write Your answer in exact form: Your answer may include fractions and/or exponents, but not decimals.

\fillwithdottedlines{2cm}
\bonusquestion[1]  The function $\displaystyle f(x)=\left (\frac{1}{2}\right )^x$ is one-to-one.
\begin{oneparchoices}
	\choice True
	\choice False
\end{oneparchoices}

\question The function $f(x)$ is given by: $\displaystyle f(x) = 2^{x+1}+2$. 

\begin{parts}
	\part[1] Write the domain of  $\displaystyle f(x)$ in interval form. \dotfill
	\part[2] Write the function transformations of $\displaystyle y=2^x$ that result in $f(x)$. Show evidence of Your thinking.
	
	\fillwithdottedlines{4cm}
	\part[1] Write the range of  $\displaystyle f(x)$ in interval form. \dotfill
	
	\part[1] As $x\to\infty$, $f(x)\to$ \dotfill
	\part[1] As $x\to-\infty$, $f(x)\to$ \dotfill
	\part[2] Find the $y$-intercept of $f(x)$. Show Your work step-by-step.
		
		\fillwithdottedlines{1cm}

	
\end{parts}
\question[1]  Find the growth factor for the situation described below. Assume that the time is in years.

{\ttfamily{Water usage is increasing by 3\% per year.}}

(Note: you need to find the growth factor only, not the equation for the population growth.)

		\fillwithdottedlines{2cm}
\end{questions}

\end{document}                 