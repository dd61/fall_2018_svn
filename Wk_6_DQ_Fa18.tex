\documentclass[12pt,dvipsnames]{article}
\usepackage[margin=0.4in,footskip=0.1in]{geometry}
\usepackage{etex}
\usepackage{amssymb,amsmath,multicol} %<-- InWorksheetExam1 i also have fancyhdr,
\usepackage{hyperref}
\usepackage[metapost]{mfpic}
\usepackage[pdftex]{graphicx}
\usepackage{csquotes}
\usepackage{pst-plot}
\usepackage{pgfplots}
\pgfplotsset{compat=1.9}

\usepackage{tikz}
\usepackage{tkz-2d}
\usepackage{tkz-base}
\usetikzlibrary{calc}
\usepackage{color}
\usepackage[inline]{enumitem}
\usepackage{refcount}%<-- non in WorksheetExam1

\usepackage[linewidth=1pt]{mdframed}

\usepackage{caption}
\usetikzlibrary{calc,fit,intersections,shapes,calc}
\usetikzlibrary{backgrounds}
\usepackage{systeme}
\usepackage{multicol}

\newcolumntype{?}{!{\vrule width 1pt}}


%%These three lines are for the typewriter font. Comment them out if I don't want the font.
%%%%%%\renewcommand*\ttdefault{lcmtt}
%%%%%%\renewcommand*\familydefault{\ttdefault} %% Only if the base font of the document is to be typewriter style
%%%%%%\usepackage[T1]{fontenc}
%%%%%%

\usepackage{tabularx, booktabs}


\newenvironment{myitemize}
{ \begin{itemize}
		\setlength{\itemsep}{10pt}
		\setlength{\parskip}{10pt}
		\setlength{\parsep}{10pt}     }
	{ \end{itemize}   
	
} 

\usepackage{setspace}

\font\maxi=cminch scaled 100
\usepackage{tgadventor}
%\renewcommand*\familydefault{\sfdefault} %% Only if the base font of the document is to be sans serif
\usepackage[T1]{fontenc}
\newcommand*{\myfont}{\fontfamily{\sfdefault}\selectfont}
\usepackage{pacioli}
\usepackage[OT1]{fontenc}
\usepackage{systeme}


%\usepackage{spalign}


%\usepackage{AlegreyaSans} %% Option 'black' gives heavier bold face
%% The 'sfdefault' option to make the base font sans serif
%\renewcommand*\oldstylenums[1]{{\AlegreyaSansOsF #1}}

\newcommand*\circled[1]{\tikz[baseline=(char.base)]{%
		\node[shape=circle,fill=blue!20,draw,inner sep=2pt] (char) {#1};}}

\usepackage{lastpage}
\usepackage{fancyhdr}
\pagestyle{fancy} 

\rfoot{{\small{Page \thepage\ of \pageref{LastPage}}}}
\cfoot{}
\renewcommand{\baselinestretch}{1.50}\normalsize



\opengraphsfile{Wk_6_DQ_Fa18}

\begin{document}
\thispagestyle{empty}

%	\thispagestyle{empty}
	\begin{center}
		{\large{Week 6}}
	\end{center}

{\bfseries{Textbook sections to read and annotate before class:}}  3.6 (skip example 2 on pg 297). This is a {\bf{difficult}} section, so You should expect to do the reading more than once to grasp the main points. For each example, please make sure that You are able to: \begin{enumerate*}[label=(\arabic*)]
  \item Find the domain;
  \item Find the $x$-intercepts, if any; \item Find the $y$-intercept, if any; \item Decide whether there is a horizontal asymptote;\item Find any vertical asymptotes.
\end{enumerate*}
In class, we will learn how to put this information together to get an idea of the shape of the graph without using a calculator.
\smallskip

	{\bfseries{Definitions to memorize before class:}} 

\begin{description}[topsep=0pt,itemsep=-2ex,partopsep=0ex,parsep=1ex]
\item[From Weeks 1-5] Linear equations and inequalities, feasible region and objective function, linear programming algorithm,  function, domain, range, interval form, transformation, operations on functions, one-to-one function, inverse, polynomial, degree, leading coefficient, end behavior, zero, multiplicity, $x\to \infty$, $x\to -\infty$. 
\item[From Section 3.6] Rational function, arrow notation from pg 296 ($x\to a^{-}, x\to a^{+}, x\to \infty, x\to -\infty$), vertical asymptote (see pg 297), horizontal asymptote (see pg 297), slant asymptote (see pg 305).
\end{description}
\smallskip	
	
	{\bfseries{Skills to review before class:} }
\begin{multicols}{2}
	\begin{enumerate}[topsep=0pt,itemsep=-2ex,partopsep=0ex,parsep=1ex]
		
                   \item Memorize the shape of the graphs of $\displaystyle y=\frac{1}{x}, y=\frac{1}{x^2}, y=\frac{1}{x^2}, y=\frac{1}{x^4}$ (see pg 166). These are the simplest rational functions that we can have: knowing their shape will provide us with a lot of information about more complex rational functions.
\item How to divide two polynomials. Please see the sidebar on pg 306 for a worked example of division of two polynomials.
\item Rational expressions (addition, subtraction multiplication, division). Your book has a nice summary of rational expressions in section 1.4: please focus on \emph{Avoiding Common Errors} on pg 42.
		
		%%%%%%%%%%%%%%%%%%
	\end{enumerate}
		
\end{multicols}
{\bfseries{Bring to class:} } A paper notebook with your annotations of the reading and your work on the questions listed below; a pen and/or a sharpened pencil and an eraser.

{\bfseries{Laptops/Phones Policy:}}  No devices in class, unless the assignment requires it.

{\bfseries{Audio-Recording:}} I will be calling people (by name) from the class roster to go over the discussion questions: to ensure everyone's privacy, please do not audio-record the class.


\begin{center}

{\large{\bfseries{Discussion Questions for Week 6} }}
\end{center}

\begin{enumerate}[label=\arabic*., leftmargin=2\parindent,
labelindent=\parindent, labelsep=*]	

	\item True or False? A rational function is the quotient of two polynomials. 
	\item What is the domain of a rational function? How do you know?
	\item The function $\displaystyle f(x)=\frac{1}{x}$ is a rational function. What are the degrees of the numerator and the denominator?
	\item Describe the shape of the graph of $\displaystyle f(x)=\frac{1}{x}$ in words.
	
	\item On pg 295, your textbook states that for the function $\displaystyle f(x)=\frac{1}{x}$, $f(x)\to \infty$ as $x\to 0^{+}$, and $f(x)\to -\infty$ as $x\to 0^{-}$. Explain these symbols in your own words, and relate them to the graph of $f(x)$.
	\item Explain the definition of horizontal and vertical asymptote in your own words.
	\item Can the graph of a rational function go through a vertical asymptote? Why? Why not?

\item This question refers to Example 3 on pg. 298-299. The rational function $r(x)$ is given by the formula: $\displaystyle r(x)=\frac{2x^2-4x+5}{(x-1)^2}.$
\begin{enumerate}
	\item Write the domain of the function in interval form.
	\item Explain why the function does not have any $x$-intercept (this is clear from Figure 5 on pg. 299, but the book does not provide an explanation.)
	\item Find the $y$-intercept.
	\item Since $r(x)$ has no $x$-intercepts, its graph is either entirely above or entirely below the $x$-axis. Explain why the graph must be above the $x$-axis (as shown in Figure 5 on pg. 299.)
\item Write the equations of the vertical asymptotes. Challenge question: without using a calculator, describe the directions of the tails of the graph of $r(x)$ near each vertical asymptote. We will answer this question in class, so You may skip it as work through the discussion questions on Your own.
	\item In the calculations to find the horizontal asymptote, the book states: \enquote{The fractional expressions $\displaystyle \frac{4}{x},\frac{5}{x^2}, \frac{2}{x}, \frac{1}{x^2}$ all approach 0 as $x\to\pm\infty$.} Explain in words why this is the case.
	
\end{enumerate}
\item This question refers to the function $\displaystyle r(x)=\frac{(2x-1)(x+4)}{(x-1)(x+2)}$, which is discussed in Example 5 on pg. 301-302.
\begin{enumerate}
\item Find the domain and write it in interval form.
\item Write the equations of the vertical asymptotes. Challenge question (as above): what is the direction of the tails of the graph near each vertical asymptote?
	\item Explain why the $x$-intercepts are $\displaystyle x=\frac{1}{2}$ and $x=-4$. Challenge question, to be discussed in class:  does the graph of $r(x)$ crosses the $x$-axis at the $x$-intercepts, or does it bounce off? Can you tell by looking at the formula for $r$, without using a calculator? [Hint: Refer to last week's notes on how the multiplicity of an $x$-intercept of a polynomial function is reflected in the graph of the function. Rational functions follow the same rule.]
\item What is the $y$-intercept?
\item Is there a horizontal asymptote? Why? Why not?
\item How many unconnected parts do You expect the graph to have? [Hint: when You wrote the domain in interval form, how many intervals did You find?]
\item Challenge question, to be discussed in class: use the information about intercepts, asymptotes and domain to sketch a (not precise) graph of $r(x)$ without using a calculator. Before class, please plot all asymptotes and intercepts, and draw the left and right tail of the graph (shape of the graph when $x\to \infty$ and $x\to -\infty$). What other information would You need to be able to sketch a (not necessarily accurate) graph for $r(x)$ without using a calculator?
	
\end{enumerate}

\item According to Your textbook (pg 305), can the graph of a rational function cross the horizontal asymptote? 

\item Another challenge question that we will discuss in class:  explain how to graph $\displaystyle r(x)=\frac{(2x-1)(x+4)}{(x-1)^2(x+2)}$ without using a calculator. Before class, please identify the domain, plot any vertical and horizontal asymptotes, and plot all the intercepts. How many unconnected parts does the graph consist of? 

\item What are the {\textcolor{red}{vertical}} asymptotes of the rational function: $\displaystyle r(x)=\frac{(x-1)x}{(x-1)(x-2)}$? [Hint: Example 8, pg 304.]

\item The function $\displaystyle r(x)=\frac{x^2}{x-2}$ does not have a horizontal asymptote (why not?).  Using Example 9 on pgs 305 and 306, write the equation of the slant asymptote, and describe the end behavior. [Note: describing the end behavior means that You should explain what happens to the graph when $x\to \infty$ and $x\to-\infty$.]
\item \label{item:eq12} Write an equation for a rational function $r(x)$ with all of the following properties:
\begin{itemize}
	\item The graph has exactly one vertical asymptote, $x=3$;
	\item The graph has exactly one $x$ intercept, $(1,0)$, and the graph bounces off the $x$-axis at $(1,0)$;
	\item The $y$ intercept is $(0,4)$.
\end{itemize}
\item Redo question \ref{item:eq12} assuming that the graph goes across the $x$-axis at $(1,0)$.
\item Write an equation for a rational function $r(x)$ with all of the following properties:
\begin{itemize}
	\item The graph has exactly one vertical asymptote, $x=3$;
	\item The graph has an $x$ intercept at $(1,0)$, and the graph bounces off the $x$-axis at $(1,0)$;
	\item The $y$ intercept is $(0,4)$.
	\item The horizontal asymptote is $y=1$.
\end{itemize}
\item (From an old exam) This problem is about the function $\displaystyle r(x)=\frac{(x-1)^2(x+1)(x+2)}{(x-2)^2(x+0.5)(x-0.5)}$.
\begin{enumerate}
	\item Write the domain in interval form).
	\item Find the $x$ intercept(s) and $y$ intercept, if they exist.
	\item 	Write the equation(s) of  the vertical asymptote(s). 
	\item Write the equation of the horizontal asymptote and explain your reasoning.
	\item (bonus question) 	Draw a possible graph for $r(x)$. In order to receive any credit for this question, you must include {\bfseries{all}} of the following items:
	\begin{enumerate}
		\item Your graph includes units both on the $x$ and the $y$ axes.
		\item You have explained how you know the behavior of the function near the vertical asymptote(s). (Do the tails go in the same direction or in opposite direction? How do you know?)
		\item If the function has $x$ intercept(s), you have stated whether the graph goes across the $x$ axis or bounces off, and how you can see this from the equation.
	\end{enumerate}
\end{enumerate}
		
\end{enumerate}
		


 \begin{mdframed}[style=exampledefault,linecolor=blue,linewidth=4pt,frametitle={Looking Ahead to EXAM DAY...}]
 	{\underline{Definitions that you should be familiar with by next Tuesday (EXAM DAY):} }
 	\begin{enumerate}[label= {  \arabic*:},labelindent=2em, style = standard,leftmargin=4pc, labelsep=*, noitemsep]
 		\item Rational function, vertical asymptote, horizontal asymptote, slant asymptote;
\item Arrow notation (from pg 296).

 	\end{enumerate}
 	{\underline{By Next Tuesday (EXAM DAY) you should be able to:} }
 	\begin{enumerate}[label= {  \arabic*:},labelindent=2em, style = standard,leftmargin=4pc, labelsep=*, noitemsep]
 		\item Find the domain and vertical asymptotes of a rational functionl
                     \item Describe the direction of the tails of the graph of a rational function near a vertical asymptote;\item Decide whether a rational function has a horizontal asymptote, a slant asymptote or no horizontal or slant asymptote;
\item Write the equation of the slant or horizontal asymptote, if it exists;
\item Describe the shape of the graph of a rational function (written in factored form) near each of the $x$-intercepts;
\item Draw a qualitative graph of a rational function, using only information derived from the formula of the function (no calculator). NOTE: I will not include graphs of functions with slant asymptotes; however, You need to be able to find any slant asymptotes using long division (see pg 306) and describe the end behavior.
 	\end{enumerate}
{\underline{Advice on how to excel in precalculus:} }
\begin{enumerate}[label= {  \arabic*:},labelindent=2em, style = standard,leftmargin=4pc, labelsep=*, noitemsep]
 		\item  Always go through the discussion questions slowly again yourself after each class, to make sure you fully understand.
Check the  work you did on the questions before class against the answers provided in class to make sure you didn't overlook anything.
\item Always read the sections we will cover{\bf{ before class}}.
\item Always attempt the discussion questionsr{\bf{ before class}}.
\item Give yourself sufficient time to do the homework: start early and don't just try to guess the answers. Work through each problem and take notes on how you attempt to solve the problem, why your attempt works or where you got stuck, and why.
\end{enumerate}
 	\end{mdframed}


\end{document} 
              