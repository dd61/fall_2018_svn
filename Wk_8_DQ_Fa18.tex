\documentclass[12pt,dvipsnames]{article}
\usepackage[margin=0.4in,footskip=0.1in]{geometry}
\usepackage{etex}
\usepackage{amssymb,amsmath,multicol} %<-- InWorksheetExam1 i also have fancyhdr,
\usepackage{hyperref}
\usepackage[metapost]{mfpic}
\usepackage[pdftex]{graphicx}
\usepackage{csquotes}
\usepackage{pst-plot}
\usepackage{pgfplots}
\pgfplotsset{compat=1.9}

\usepackage{tikz}
\usepackage{tkz-2d}
\usepackage{tkz-base}
\usetikzlibrary{calc}
\usepackage{color}
\usepackage[inline]{enumitem}
\usepackage{refcount}%<-- non in WorksheetExam1

%\usepackage[linewidth=1pt]{mdframed}


\usepackage[framemethod=TikZ]{mdframed}
\newcommand{\mdfLABEL}[1]{\node[text width=2em,align=right,anchor=north east,%
	outer sep=0pt,inner sep=0pt] at ($ (O|-P)
	-(\the\mdflength{innerleftmargin},0)
	-0.5*(\the\mdflength{middlelinewidth},0)
	- (0,\the\mdflength{innertopmargin})
	+ (0,0.5pt)
	$) {$\Box$};} %in the original code, $F$ is replaced with #1

\mdfdefinestyle{testframe}{topline=false,rightline=false,bottomline=false,%
	innerleftmargin=1em,linecolor=white,rightmargin=2em,skipbelow=1em,%
	tikzsetting={draw=black,line width=.5pt,dashed,dash pattern= on 1pt off 3pt},%
	firstextra={\mdfLABEL{(i)}},%
	singleextra={\mdfLABEL{(i)}},%
	secondextra={\mdfLABEL{$\phantom{.}$}},%
	middleextra={\mdfLABEL{$\phantom{.}$}},%
}



\usepackage{caption}
\usetikzlibrary{calc,fit,intersections,shapes,calc}
\usetikzlibrary{backgrounds}
\usepackage{systeme}
\usepackage{multicol}

\newcolumntype{?}{!{\vrule width 1pt}}


%%These three lines are for the typewriter font. Comment them out if I don't want the font.
%%%%%%\renewcommand*\ttdefault{lcmtt}
%%%%%%\renewcommand*\familydefault{\ttdefault} %% Only if the base font of the document is to be typewriter style
%%%%%%\usepackage[T1]{fontenc}
%%%%%%

\usepackage{tabularx, booktabs}


\newenvironment{myitemize}
{ \begin{itemize}
		\setlength{\itemsep}{10pt}
		\setlength{\parskip}{10pt}
		\setlength{\parsep}{10pt}     }
	{ \end{itemize}   
	
} 

\usepackage{setspace}

\font\maxi=cminch scaled 100
\usepackage{tgadventor}
%\renewcommand*\familydefault{\sfdefault} %% Only if the base font of the document is to be sans serif
\usepackage[T1]{fontenc}
\newcommand*{\myfont}{\fontfamily{\sfdefault}\selectfont}
\usepackage{pacioli}
\usepackage[OT1]{fontenc}
\usepackage{systeme}


%\usepackage{spalign}


%\usepackage{AlegreyaSans} %% Option 'black' gives heavier bold face
%% The 'sfdefault' option to make the base font sans serif
%\renewcommand*\oldstylenums[1]{{\AlegreyaSansOsF #1}}

\newcommand*\circled[1]{\tikz[baseline=(char.base)]{%
		\node[shape=circle,fill=blue!20,draw,inner sep=2pt] (char) {#1};}}

\usepackage{lastpage}
\usepackage{fancyhdr}
\pagestyle{fancy} 

\rfoot{{\small{Page \thepage\ of \pageref{LastPage}}}}
\cfoot{}
\renewcommand{\baselinestretch}{1.50}\normalsize




\makeatletter
\newenvironment{enumeratecount}[1]
{\def\thisenumeratecountlabel{#1}\enumerate}
{\edef\@currentlabel{\number\value{\@enumctr}}%
	\label{\thisenumeratecountlabel}\endenumerate}
\makeatother

\usepackage{refcount}
\newcommand{\addphrase}[3]{% #1 = label, #2 = text if number >1, #3 = text if number =1
	\ifnum\getrefnumber{#1}>1
	#2%
	\else
	#3%
	\fi}


\makeatletter
% This command ignores the optional argument for itemize and enumerate lists
\newcommand{\inlineitem}[1][]{%
	\ifnum\enit@type=\tw@
	{\descriptionlabel{#1}}
	\hspace{\labelsep}%
	\else
	\ifnum\enit@type=\z@
	\refstepcounter{\@listctr}\fi
	\quad\@itemlabel\hspace{\labelsep}%
	\fi}
\makeatother

\newcommand*\circledA[1]{\tikz[baseline=(char.base)]{%
		\node[shape=circle,fill=green!20,draw,inner sep=2pt] (char) {#1};}}

\opengraphsfile{Wk_8_DQ_Fa18}

\begin{document}
\thispagestyle{empty}

%	\thispagestyle{empty}
	\begin{center}
		{\large{Week 8}}
	\end{center}

{\bfseries{Textbook sections to read and annotate before class:}} 4.1 and 4.2.
%\begin{enumerate*}[label=(\arabic*)]

\smallskip

	{\bfseries{Definitions to memorize before class:}} 

\begin{description}[topsep=0pt,itemsep=-2ex,partopsep=0ex,parsep=1ex]
\item[From Weeks 1-6] Linear equations and inequalities, feasible region and objective function, linear programming algorithm,  function, domain, range, interval form, transformation, operations on functions, one-to-one function, inverse, polynomial, degree, leading coefficient, end behavior, zero, multiplicity, $x\to \infty$, $x\to -\infty$, rational function, arrow notation from pg 296 ($x\to a^{-}, x\to a^{+}, x\to \infty, x\to -\infty$), vertical asymptote, horizontal asymptote, slant asymptote. 
\item[From Sections 4.1, 4.2] Exponential function, graph of an exponential function, compound interest, annual percentage yield, $e$, natural exponential function, continuous compounding.
\end{description}
\smallskip	
	
	{\bfseries{Skills to review before class:} }
\begin{multicols}{2}
	\begin{enumerate}[topsep=0pt,itemsep=-2ex,partopsep=0ex,parsep=1ex]
		
\item How to change a negative exponent into a positive exponent;
\item How to change a radical exponent into a root;
\item How to apply the order of operations rule;
\item How to identify the domain, range, asymptotes of a function;
\item How to apply the vertical line test and the horizontal line test;
\item How to decide whether a function is linear, quadratic, polynomial or rational.
		
		%%%%%%%%%%%%%%%%%%
	\end{enumerate}
		
\end{multicols}
{\bfseries{Bring to class:} } A paper notebook with your annotations of the reading and your work on the questions listed below; a pen and/or a sharpened pencil and an eraser.

{\bfseries{Laptops/Phones Policy:}}  No devices in class, unless the assignment requires it.

{\bfseries{Audio-Recording:}} I will be calling people (by name) from the class roster to go over the discussion questions: to ensure everyone's privacy, please do not audio-record the class.


\begin{center}

{\large{\bfseries{Discussion Questions for Week 8} }}
\end{center}

\begin{mdframed}[style=testframe]
	An exponential is a function defined as $\displaystyle f(x)=a^x$, where $a$ is a fixed positive number different from 1, and $x$ spans over all real numbers (so the domain of $\displaystyle f(x)=a^x$ is $\displaystyle (-\infty,\infty )$.) The number $a$ is called the {\emph{base}} of the exponential function.
\end{mdframed}


\begin{enumerate}[label=\protect\circled{\arabic*}]
	%%%%%%%%%%%%%%%%%%%
	%%%%% https://www.illustrativemathematics.org/HSF-LE.A
	%%%%% GREAT REFERENCE
	%%%%%%%%%%%%%%%%%%%
	\item If $\displaystyle f(x)=a^x$, why can't $a$ be 1? Why do we require that $a\not = 0$? Why can't $a$ be negative?
	\item  In the reading, you have seen the graphs of $\displaystyle f(x)=a^x$ when $a>1$ and when $0<a<1$. Describe the graph of $\displaystyle f(x)=a^x$ when $a>1$ in words. What are the similarities and differences with the graph of $\displaystyle f(x)=a^x$ when $0<a<1$? [Words to use in your descriptions: domain, range, asymptotes, intercepts, left tail, right tail.]
	
	
	
\item What function transformation of $\displaystyle f(x)=3^x$ results in $\displaystyle g(x)=\left ( \frac{1}{3}\right )^x$? (Hint: Example 2 in section 4.1.)

\item What function transformation of $\displaystyle f(x)=10^x$ results in $\displaystyle g(x)=3-10^{x-1}$? In what order should the transformations be applied?

\item How does the graph of $\displaystyle f(x)=9^{\frac{x}{2}}$ relate to the graph of $\displaystyle g(x)=3^x$? (Hint: use the properties of exponents.)

\item Write the equation of an exponential function $\displaystyle f(x)=a^x$, where the point $(-3,8)$ is on the graph of the exponential.
	
	\item A piece of paper has a thickness of 0.1 mm (note that 1 mm is one-thousandth of a meter). If we fold the piece of paper 7 times, how thick is the paper? How about if we fold 12 times? How many times should we fold the paper so that the resulting stack is one kilometer thick? (Write a formula, do the calculations, then watch this video \url{https://www.youtube.com/watch?v=AAwabyyqWK0} for the answer. Another great video on folding is \url{https://www.youtube.com/watch?v=kRAEBbotuIE}).
	
	
\end{enumerate}
\begin{mdframed}[style=testframe]
	The difference between linear growth and exponential growth is that with linear growth, a quantity increases or decreases by a constant number, while in exponential growth, a quantity increases or decreases by a constant percent. The base $a$ of the exponential $\displaystyle y=a^x$ is also called the growth factor.  The growth factor is given by $a=1+r$ (if the exponential increases) or $a=1-r$ (if the exponential decreases), where $r$ is the decimal representation of the percent rate of change.
\end{mdframed}

\begin{enumerate}[label=\protect\circled{\arabic*},resume]
	\item Find the growth factor for each of the \ref{firstlist}~\addphrase{firstlist}{situations}{situation}  described below. Assume that the time is in the units indicated in each exercise. %The following list contains \ref{firstlist}~\addphrase{firstlist}{items}{item}
	\begin{enumeratecount}{firstlist}
		\item The population of a city grows by 5\% per decade.
		\item Water usage is increasing by 3\% per year.
		\item After we drink a cup of coffee, every hour approximately 15\% of the amount of caffeine in the body is metabolized and eliminated.
		%\end{enumerate}
	\end{enumeratecount}
	\item The population $P$, of six towns (with time $t$ in years) is given by:
	\begin{enumeratecount}{secondlist}
		\item $\displaystyle P=1000(1.08)^t$ \inlineitem $\displaystyle P=600(1.12)^t$\inlineitem $\displaystyle P=2500(0.9)^t$
		\item $\displaystyle P=1200(1.186)^t$ \inlineitem $\displaystyle P=800(0.78)^t$\inlineitem $\displaystyle P=2000(0.99)^t$
	\end{enumeratecount}
	\begin{enumerate}[label=\fbox{Part\arabic{*}}, ref=(Part\arabic{*}),leftmargin=5.0em]
		\item Which of the \ref{secondlist}~\addphrase{secondlist}{towns}{town} are growing in size? Which are shrinking?
		\item Which of the \ref{secondlist}~\addphrase{secondlist}{towns}{town} is growing the fastest? By What percent is this town's population growing each year?
		\item Which of the \ref{secondlist}~\addphrase{secondlist}{towns}{town} is decreasing the fastest? By What percent is this town's population decreasing each year?
		\item Which of the \ref{secondlist}~\addphrase{secondlist}{towns}{town} has the largest initial population (at $t=0$)? Which town has the smallest initial population?
	\end{enumerate}

	
%%%	\item This graph \url{https://www.statista.com/chart/7581/amazons-global-workforce/} shows Amazon's global workforce over time. Does the growth appear exponential? Is it exponential? How about Amazon's net sales revenue from 2004 to 2017? Check the graph at \url{https://www.statista.com/statistics/266282/annual-net-revenue-of-amazoncom/}.
	
	
	\item A town's population increases in one year from 100,000 people to 100,100 people.
	\begin{enumerate}
		\item If the population grows linearly, what is the population size after 20 years?
		\item if the population grows exponentially, what is the population size after 20 years?
	\end{enumerate}
	\item For each of the following items, decide whether the quantity is changing linearly or exponentially.
	\begin{enumerate}[label=\protect\circledA{\arabic*}]
		\item 	A savings account, which earns no interest, receives a deposit of \$300 per month.
		\item The value of a device depreciates by 10\% per year.
		\item Each Monday, the amount of a radioactive substance is 7/10 of the amount on the previous Monday.
		\item A patient takes a dosage of a medication. Every four hours the amount of medication in the patient's body decreases by half.
	\end{enumerate}
	

	
	
	\item Suppose a single bacterium is placed in a bottle at 11:00 am. It grows and at 11:01 am it divides into two bacteria. The two bacteria grow and at 11:02 am each of them divides into two bacteria, and so on. The bacteria continues to double every minute, and the bottle is full at noon.
	
	\begin{enumerate}[label=\protect\circledA{\arabic*}]
		\item How many more bacteria are in the bottle at 11:31 am than at 11:30 am? How do you know?
		\item  At what time is  the bottle half-full? 
		\item  It is 11:56 am. What fraction of the bottle is full at this time?
	\end{enumerate}
	
\end{enumerate}


\begin{mdframed}[style=testframe]
	\begin{itemize}
		\item[$\circ$] If an initial principal $P$ is invested at an annual rate
		$r$ and the interest is compounded $n$ times per year, the amount $A$ in the account after $t$ years is
		$\displaystyle A(t) = P\left ( 	1 + \frac{r}{n} \right ) ^{nt}$. The interest $r$ is called the APR (annual percentage rate); $r$ is also called the nominal rate.
		
		\item[$\circ$] If If an initial principal $P$ is invested at an annual rate
		$r$ and the interest is compounded continuously (that is, the interest is compounded at every instant), then the the amount $A$ in the account after $t$ years is
		$\displaystyle A(t) = Pe^{rt}$, where $e$ is a constant number ($e$ is approximately 2.718).
		
		\item[$\circ$] The present value of a sum of money is the amount that must be invested today, at a given interest rate, to yield the desired amount at a later time.
		
		\item[$\circ$] If an investment earns compound interest, then the APY (annual percentage yield) is the simple interest that earns the same amount of money at the end of one year. The APY equals: $\displaystyle \left ( 	1 + \frac{r}{n} \right ) ^{n}-1$.
	\end{itemize}
	
\end{mdframed}

\begin{enumerate}[label=\protect\circled{\arabic*},resume]	


\item If \$10,000 is deposited in an account that pays interest at an annual rate of 0.3\%, write a formula for the amount in the account 5 years after the investment (assuming that no money is withdrawn, and no money other than the interest is deposited) if the interest is compounded:
\begin{multicols}{2}
	\begin{enumerate}[topsep=0pt,itemsep=-2ex,partopsep=0ex,parsep=1ex]
%\begin{enumerate}
	\item semiannually;
	\item quarterly;
	\item daily;
	\item continuously.
\end{enumerate}	
\end{multicols}
\item An investment grows by 1\% per year for 10 years. By what percent does it increase over the 10-year period?
	\item An investment earns 1\% per year, compounded monthly. Find the APY for this investment.
\item A grandmother plans to invest in a college savings plan for a period of 18 years: the plan will pay an interest of 2\% compounded semiannually, and the grandmother would like the plan to grow to \$40,000 after 18 years. How much will she need to invest?
		
\end{enumerate}


		
{\bf More problems on exponential functions}
\begin{enumerate}[label=\protect\circled{\arabic*},resume]
	\item The population of an endangered species is decreasing by 10\% per year. How many years will it take the population to drop by half?
	
	\item A population is growing by 10\% per month. By what factor will it grow in 1 year?
	
	\item A population doubles every 4 hours. By what factor does it grow in 12 hours?
	
	\item By which percent will each of the \ref{firstlist}~\addphrase{firstlist}{quantities}{quantity}  described below increase in one year?
	\begin{enumeratecount}{thirdlist}
		\item Doubles in size every 7 years;
		\item Triples in size every 7 years;
		\item Grows by 3\% per month;
		\item Grows by 10\% every 4 months.
	\end{enumeratecount}

	\item The video \url{https://youtu.be/PUwmA3Q0_OE} by the American Museum of Natural History shows Earth's population growth. Does the growth appear exponential? According to the video, how long did it take for the population to reach one billion? How many more years did it take for the population to reach 7 billion?

\end{enumerate}
%https://youtu.be/PUwmA3Q0_OE		


 \begin{mdframed}[style=exampledefault,linecolor=blue,linewidth=4pt,frametitle={Looking Ahead to the second half of the course...}]

{\underline{Advice on how to excel in precalculus:} }
\begin{enumerate}[label= {  \arabic*:},labelindent=2em, style = standard,leftmargin=4pc, labelsep=*, noitemsep]
 		\item  Always go through the discussion questions slowly again yourself after each class, to make sure you fully understand.
Check the  work you did on the questions before class against the answers provided in class to make sure you didn't overlook anything.
\item Always read the sections we will cover{\bf{ before class}}.
\item Always attempt the discussion questionsr{\bf{ before class}}.
\item Give yourself sufficient time to do the homework: start early and don't just try to guess the answers. Work through each problem and take notes on how you attempt to solve the problem, why your attempt works or where you got stuck, and why.
\end{enumerate}
 	\end{mdframed}


\end{document} 
              