\documentclass[12pt,dvipsnames]{article}
\usepackage[margin=0.4in,footskip=0.1in]{geometry}
\usepackage{etex}
\usepackage{amssymb,amsmath,multicol} %<-- InWorksheetExam1 i also have fancyhdr,
\usepackage{hyperref}
\usepackage[metapost]{mfpic}
\usepackage[pdftex]{graphicx}
\usepackage{csquotes}
\usepackage{pst-plot}
\usepackage{pgfplots}
\pgfplotsset{compat=1.9}

\usepackage{tikz}
\usepackage{tkz-2d}
\usepackage{tkz-base}
\usetikzlibrary{calc}
\usepackage{color}
\usepackage[inline]{enumitem}
\usepackage{refcount}%<-- non in WorksheetExam1

%\usepackage[linewidth=1pt]{mdframed}


\usepackage[framemethod=TikZ]{mdframed}
\newcommand{\mdfLABEL}[1]{\node[text width=2em,align=right,anchor=north east,%
	outer sep=0pt,inner sep=0pt] at ($ (O|-P)
	-(\the\mdflength{innerleftmargin},0)
	-0.5*(\the\mdflength{middlelinewidth},0)
	- (0,\the\mdflength{innertopmargin})
	+ (0,0.5pt)
	$) {$\Box$};} %in the original code, $F$ is replaced with #1

\mdfdefinestyle{testframe}{topline=false,rightline=false,bottomline=false,%
	innerleftmargin=1em,linecolor=white,rightmargin=2em,skipbelow=1em,%
	tikzsetting={draw=black,line width=.5pt,dashed,dash pattern= on 1pt off 3pt},%
	firstextra={\mdfLABEL{(i)}},%
	singleextra={\mdfLABEL{(i)}},%
	secondextra={\mdfLABEL{$\phantom{.}$}},%
	middleextra={\mdfLABEL{$\phantom{.}$}},%
}



\usepackage{caption}
\usetikzlibrary{calc,fit,intersections,shapes,calc}
\usetikzlibrary{backgrounds}
\usepackage{systeme}
\usepackage{multicol}

\newcolumntype{?}{!{\vrule width 1pt}}


%%These three lines are for the typewriter font. Comment them out if I don't want the font.
%%%%%%\renewcommand*\ttdefault{lcmtt}
%%%%%%\renewcommand*\familydefault{\ttdefault} %% Only if the base font of the document is to be typewriter style
%%%%%%\usepackage[T1]{fontenc}
%%%%%%

\usepackage{tabularx, booktabs}


\newenvironment{myitemize}
{ \begin{itemize}
		\setlength{\itemsep}{10pt}
		\setlength{\parskip}{10pt}
		\setlength{\parsep}{10pt}     }
	{ \end{itemize}   
	
} 

\usepackage{setspace}

\font\maxi=cminch scaled 100
\usepackage{tgadventor}
%\renewcommand*\familydefault{\sfdefault} %% Only if the base font of the document is to be sans serif
\usepackage[T1]{fontenc}
\newcommand*{\myfont}{\fontfamily{\sfdefault}\selectfont}
\usepackage{pacioli}
\usepackage[OT1]{fontenc}
\usepackage{systeme}
%\usepackage{ulem}

%\usepackage{spalign}


%\usepackage{AlegreyaSans} %% Option 'black' gives heavier bold face
%% The 'sfdefault' option to make the base font sans serif
%\renewcommand*\oldstylenums[1]{{\AlegreyaSansOsF #1}}

\newcommand*\circled[1]{\tikz[baseline=(char.base)]{%
		\node[shape=circle,fill=blue!20,draw,inner sep=2pt] (char) {#1};}}

\usepackage{lastpage}
\usepackage{fancyhdr}
\pagestyle{fancy} 

\rfoot{{\small{Page \thepage\ of \pageref{LastPage}}}}
\cfoot{}
\renewcommand{\baselinestretch}{1.50}\normalsize




\makeatletter
\newenvironment{enumeratecount}[1]
{\def\thisenumeratecountlabel{#1}\enumerate}
{\edef\@currentlabel{\number\value{\@enumctr}}%
	\label{\thisenumeratecountlabel}\endenumerate}
\makeatother

\usepackage{refcount}
\newcommand{\addphrase}[3]{% #1 = label, #2 = text if number >1, #3 = text if number =1
	\ifnum\getrefnumber{#1}>1
	#2%
	\else
	#3%
	\fi}


\makeatletter
% This command ignores the optional argument for itemize and enumerate lists
\newcommand{\inlineitem}[1][]{%
	\ifnum\enit@type=\tw@
	{\descriptionlabel{#1}}
	\hspace{\labelsep}%
	\else
	\ifnum\enit@type=\z@
	\refstepcounter{\@listctr}\fi
	\quad\@itemlabel\hspace{\labelsep}%
	\fi}
\makeatother

\newcommand*\circledA[1]{\tikz[baseline=(char.base)]{%
		\node[shape=circle,fill=green!20,draw,inner sep=2pt] (char) {#1};}}

\opengraphsfile{Wk_11_DQ_Fa18}

\begin{document}
\thispagestyle{empty}

%	\thispagestyle{empty}
	\begin{center}
		{\large{Week 11}}
	\end{center}

{\bfseries{Textbook sections to read and annotate before class:}} 5.1, 5.2, 5.3 (up to pg 424. We will talk about shifted sine and cosine curves next week).
%\begin{enumerate*}[label=(\arabic*)]

\smallskip

	{\bfseries{Definitions to memorize before class:}} 

\begin{description}[topsep=0pt,itemsep=-2ex,partopsep=0ex,parsep=1ex]
\item[From Weeks 1-10] Linear equations and inequalities, feasible region and objective function, linear programming algorithm,  function, domain, range, interval form, transformation, operations on functions, one-to-one function, inverse, polynomial, degree, leading coefficient, end behavior, zero, multiplicity, $x\to \infty$, $x\to -\infty$, rational function, arrow notation from pg 296 ($x\to a^{-}, x\to a^{+}, x\to \infty, x\to -\infty$), vertical asymptote, horizontal asymptote, slant asymptote, exponential function, graph of an exponential function, compound interest, annual percentage yield, $e$, natural exponential function, continuous compounding, $\log_a x$, $\log x$, $\ln x$, doubling time, half life, relative growth rate.
\item[From Week 11] Unit circle, terminal point, reference point, periodic function, sine, cosine, amplitude, period.
\end{description}
\smallskip	
	
	%%{\bfseries{Skills to review before class:} }
%%\begin{multicols}{2}
	%\begin{enumerate}[topsep=0pt,itemsep=-2ex,partopsep=0ex,parsep=1ex]
		
%%	\begin{enumerate}[label= {  \arabic*:},labelindent=1em, style = standard,leftmargin=3pc, labelsep=*, itemsep=-2ex,partopsep=0ex,parsep=1ex]
%%		\item Rewrite a log identity or equation in exponential form.
%%		\item Rewrite a exponential identity or equation in log form.
%%		\item Find the domain, range,  vertical asymptote and intercept(s) of a transformation of a logarithmic function.
%%		\item Expand or condense an expression using the laws of logarithms.
		
		%%%%%%%%%%%%%%%%%%
%%	\end{enumerate}
		
%\end{multicols}
{\bfseries{Bring to class:} } A paper notebook with your annotations of the reading and your work on the questions listed below; a pen and/or a sharpened pencil and an eraser.

{\bfseries{Laptops/Phones Policy:}}  No devices in class, unless the assignment requires it.

{\bfseries{Audio-Recording:}} I will be calling people (by name) from the class roster to go over the discussion questions: to ensure everyone's privacy, please do not audio-record the class.


\begin{center}

{\large{\bfseries{Discussion Questions for Week 11} }}
\end{center}
	%\begin{enumerate}[label=\protect\circled{\arabic*}]
		\renewcommand{\labelenumi}{(\arabic{enumi})}
		%%%%%%%%%%%%%%%%%%%
		%%%%% https://www.illustrativemathematics.org/HSF-LE.A
		%%%%% GREAT REFERENCE
		%%%%%%%%%%%%%%%%%%%

\begin{enumerate}[label= {\bf  \arabic*:}]
	\item A Ferris wheel with 59 capsules has a diameter of 110 meters and its lowest point (at the 6 o'clock position) is 5 meters above the ground. One full rotation of the wheel takes 24 minutes, and the wheel rotates counterclockwise at a constant speed. The capsules are labeled from 1 to 59. We start observing when capsule 1 is at the lowest point on the wheel. 
	\begin{enumerate}
		\item Let $H(t)$ be the height of capsule 1 $t$ minutes after we start the observation. Fill out this table:
		
		\begin{minipage}{\linewidth}
			\centering
			%\captionof{table}{} %\label{tab:title} 
			\begin{tabularx}{0.8\textwidth}{|X|X|X|X|X|X|X|X|X|X|X|X|X|}
				\hline
				\multicolumn{2}{|c|}{$t$}         &0&6& 12 & 18 & 24 & 30 & 36 & 42 &48&2&3\\ \hline
				\multicolumn{2}{|c|}{$H(t)$}   & & &     &     &     &     &     && & &     \\ \hline
			\end{tabularx}
		\end{minipage}
		
		\item Draw the graph of $H(t)$ over two rotations of the wheel.
		\item The function $H(t)$ is periodic because it repeats itself every 24 minutes. More generally, if a function $f(t)$ completes one full cycle in a time interval of length $c$, then $c$ is called the period of this function (note that the period is the smallest time interval in which the function completes one cycle). The midline of a periodic function is the horizontal line that lies halfway between the peaks and the valleys of the graph. The amplitude of a periodic function is half of the vertical distance between the peaks and the valleys. What are the midline and the amplitude of the function $H(t)$?
	%	\item (Optional) During the first rotation, for how many minutes is capsule 1 at a height of at least 65 meters from the ground?
	%	\item How would the graph change if we assume that the lowest point of the wheel is 4 meters above the ground?
	%	\item \label{item:4oclock} How would the graph change if we assume that we start observing when capsule 1 is at the 4 o'clock position? 
	%	\item How would the graph in Question \ref{item:4oclock} change if  the wheel rotates clockwise?
		
	\end{enumerate}
	
	
	\item The unit circle is the circle with radius 1 and center at $(0,0)$: its equation is $\displaystyle x^2+y^2=1$ (why??) 
 Are these points on the unit circle? $(1,0), (0,1), (2,0), \left( \frac{\sqrt{3}}{2},\frac{1}{2}\right),
		% \left(-\frac{\sqrt{3}}{2},\frac{1}{2}\right), 
		(-1,0), \left(\frac{\sqrt{3}}{2},-\frac{1}{2}\right)
		$.

	
	\item The $x$ and $y$ axes divide the unit circle into four parts. (For example, one of the four parts is in the top right quadrant, starting at $(1,0)$ and ending at $(0,1)$.) What is the length of each part?
	
	\item \label{ref:partetre} Now we visualize numbers using the unit circle. To visualize a number, imagine wrapping a measuring tape along the circle, using these rules:
	
	
	\begin{mdframed}[style=MyFrame]
		\begin{enumerate}[label=\protect\circled{\arabic*}]
			\item  Your measuring tape has negative numbers, positive numbers and zero;
			\item \label{itemB} The zero tick mark on the tape is place at the point $(1,0)$ on the unit circle, and the tick mark 1 on the tape is on the top right quadrant of the circle;
			Your measuring tape has marks for integers as well as non-integer numbers. For example, the tape has marks for $\pi, 2\pi, 3\pi, -\pi, -2\pi, -3\pi, \frac{\pi}{4}, \frac{\pi}{3}, \frac{\pi}{2},-\frac{\pi}{4},$ $-\frac{\pi}{3}$, $-\frac{\pi}{2},$  $\frac{7\pi}{4},$ $-\frac{7\pi}{4}, \ldots$. 
		\end{enumerate}
	\end{mdframed}
	
	
	\begin{enumerate}
		\item Visualize the numbers in \ref{itemB} on the unit circle. The point on the unit circle that represents each number is called the terminal point.
		\item Visualize the following numbers on the unit circle and give evidence of Your thinking.
		
				\begin{multicols}{5}
					\begin{enumerate}
						\item 1;
						\item 2;
						\item 3;
						\item 4;
						\item 5;
						\item -2.
						
					\end{enumerate}	
				\end{multicols}
			%\end{enumerate}	
	\end{enumerate}
	
	\item Find a number corresponding to the terminal point $(0,-1)$. Then, find two more numbers with the same terminal point.
\end{enumerate}
\begin{mdframed}[style=testframe]
	\begin{itemize}
		\item[$\circ$] The reference number $\overline{t}$ of a number $t$ is the shortest distance along the unit circle between the terminal point determined by $t$ and the $x$-axis.
		
			\item[$\circ$]  To find the terminal point of a number $t$:
			\begin{enumerate}
				\item Find the reference number $\overline{t}$  of $t$;
				\item Find the terminal point $Q(a,b)$ of $\overline{t}$ (that is, the $x$-coordinate of $Q$ is $a$ and the $y$-coordinate of $Q$ is $b$.)
				\item The $x$-coordinate of the terminal point of $t$ is $\pm a$, where You choose the positive or negative sign depending on the quadrant in which the terminal point of $t$ lies;
					\item The $y$-coordinate of the terminal point of $t$ is $\pm b$, where You choose the positive or negative sign depending on the quadrant in which the terminal point of $t$ lies.
			\end{enumerate}
		
	\end{itemize}
	
\end{mdframed}

\begin{enumerate}[label=\protect\circled{\arabic*},resume]
	\item Find the reference number of: $\displaystyle -\frac{\pi}{4},$ $\displaystyle -\frac{\pi}{3}$, $\displaystyle -\frac{\pi}{2},$  $\displaystyle \frac{7\pi}{4},$ $\displaystyle -\frac{7\pi}{4}$. Then, use the reference numbers to find the terminal points.
	
	\item The sine of a number is the $y$ coordinate of its terminal point on the unit circle. The cosine of a number is the $x$ coordinate coordinate of the terminal point. 
	\begin{enumerate}
		\item Why is it true that $\displaystyle \sin^2t + \cos^2t=1$? 
		\item Find the sine and cosine of the numbers in \ref{itemB}. 
		\item Without using a calculator, rank the following numbers from smallest to largest:
		\begin{enumerate}
			\item $\cos \left(\frac{2\pi}{3}\right ), \cos (2.3), \cos \left (\frac{2}{3}\right ), \cos \left (-\frac{2\pi}{3}\right )$;
						\item $\sin \left(\frac{2\pi}{3}\right ), \sin (2.3), \sin \left (\frac{2}{3}\right ), \sin \left (-\frac{2\pi}{3}\right )$.
		\end{enumerate}
	\end{enumerate}
	
	\item The function $\displaystyle h(t)=\sin t$ pairs each number $t$ with the $y$ coordinate of its terminal point on the unit circle. 
	\begin{enumerate}
		\item \label{ref:item71} Draw $\displaystyle h(t)=\sin t$ using inputs in $[0,2\pi]$, then using inputs in $[2\pi,4\pi]$, then using inputs in $[-2\pi, 0]$. Describe the shape of the graph (What is the range? Where are the peaks and valleys of the graph?)
		\item \label{ref:item72} Is the function $\displaystyle h(t)=\sin t$ one-to-one? Why? Why not? What is the domain?
		\item Redo parts \ref{ref:item71} and \ref{ref:item72} using $\displaystyle g(t)=\cos t$ instead of $\displaystyle h(t)=\sin t$.
		\item List the transformation(s) of $g(t)$ that result in $h(t)$.
		\item List the transformation(s) of $h(t)$ that result in $g(t)$.
	\end{enumerate}
%%%%%%%%%%%%%%%%%%%%%	
	
\end{enumerate}
\begin{mdframed}[style=testframe]
	\begin{itemize}
		\item[$\circ$] A function $f$ is periodic if there is a positive number $p$ such that $f(t+p)=f(t)$ for every $t$. The least such positive number (if it exists) is the period of $f$.
		\item[$\circ$] The functions $\displaystyle f(x)=\sin (x), g(x)=\cos (x)$ are periodic and their period is $2\pi$.
		\item[$\circ$] The amplitude of a function transformation of $\displaystyle f(x)=\sin (x)$ or  $\displaystyle g(x)=\cos (x)$ 	 is half the vertical distance between maxima and minima.
		\item[$\circ$] The midline of a function transformation of $\displaystyle f(x)=\sin (x)$ or $\displaystyle g(x)=\cos (x)$ is the line $y=D$, 
		%\end{enumerate}
		
	\end{itemize}
	
\end{mdframed}

\begin{enumerate}[label=\protect\circled{\arabic*},resume]	
	
	%%%%%%%%%%%%%%%%%
	
%	\item The textbook (on page 419) states that \enquote{A function $f$ is periodic if there is a positive number $p$ such that $f(t+p)=f(t)$ for every $t$. The least such positive number (if it exists) is the period of $f$.} 
	
	%\begin{enumerate}
		%\item Explain why the period of $\displaystyle h(t)=\sin t$ and $\displaystyle g(t)=\cos t$ is $2\pi$.
		%\item $y=\sin(2t)$ is the horizontal compression of $h(t)$ by a factor of 2. What is the period of this function?

		\item \label{item:list} List the transformations of $h(t)$ that result in each of the functions listed below. Find the period, domain, range, amplitude and midline of each of these functions.
		
		 \begin{multicols}{3}
		\begin{enumerate}[label=\fbox{\arabic*}]
			\item $\displaystyle y=-\sin t$;
			\item $\displaystyle y=\sin(-t)$;
			\item $\displaystyle y=\sin(2t)$;
			\item $\displaystyle y=\sin \left (\frac{t}{2}\right )$;
			\item $\displaystyle y=1+\sin t$;
			\item $\displaystyle y=1-\sin t$;
  	        %\item $\displaystyle y=\sin(2t)$;
			\item $\displaystyle y=\sin\left(\frac{t}{2}\right)$;
			\item $\displaystyle y=2\sin t$;
			\item $\displaystyle y=\frac{\sin t}{2}$;
			\item $\displaystyle y=\sin\left(t+\frac{\pi}{2}\right)$;
			\item $\displaystyle y=\sin\left(t-\frac{\pi}{2}\right)$.
		\end{enumerate}
		\end{multicols}
%	\item 	 (Optional) Play with the demo \url{https://www.desmos.com/calculator/xjx15igpsf}. Use it to plot some of the functions in Question \ref{item:list}, as well as functions from the list of examples in the textbook.	
%	\end{enumerate}
	\item \label{vertical}  Part of the graph of a perioding function $y=T(x)$ with domain all real numbers is shown below. The maximum and minimum values of $T(x)$ are shown. 
	
  
		\begin{center}
			
			
			
			\begin{mfpic}[20]{-2}{3}{-2}{3}
				
				
				\polyline{(-2,1), (-1,-2)} 
				
				\polyline{(-1,-2), (0,0)}
				
				\polyline{(0,0),(1,-2)}
				
				\polyline{(1,-2),(2,1)}
				
				\polyline{(2,1),(3,-2)}
				
				%\point[5pt]{(0,2),  (2,2), (2,1),(5,3)}
				%\circle{(5,1),0.15}
				%\circle{(2,2),0.15}
				\axes
				
				\xmarks{-2,-1,,1,2,3}
				
				\ymarks{-2,-1,,1,2,3}
				
				\tlpointsep{4pt}
				
				\axislabels {x}{{\tiny $-2$} -2, {\tiny $-1$} -1, {\tiny $0$} 0, {\tiny $1$} 1, {\tiny $2$} 2}
				
				\axislabels {y}{{\tiny $1$} 1,{\tiny $2$} 2, {\tiny $3$} 3,  {\tiny $-1$} -1, {\tiny $-2$} -2}
				\tlabels{[tc](\xmax, 0){$x$} [cr](0, \ymax){$T(x)$}}
				% Grid
				%\drawcolor[gray]{0.25}
				%\gridlines{1, 1}
				\drawcolor[gray]{0.75} 
				\grid{1,1}
				
				
				
				
			\end{mfpic}
			
		\end{center}
\begin{enumerate}
\item Which of these numbers can be the period of the function $T$?
		 \begin{multicols}{3}
		 \begin{enumerate}[label=\fbox{\arabic*}]
		 \item 1
		 \item 2
		 \item 4
		 \item 5
		 \item 6
		 \item None of these.
		 \end{enumerate}
		 \end{multicols}	

	\item Write the equation of the midline of $T$, and find the amplitude of the function.
\end{enumerate}	

\item True or False? Show evidenc of Your thinking.

If $f(x)$ is a periodic function and its period is 5, then $f(1)=f(11)$.

\item The height (in cm) of the tip of the hour hand on a vertical clock face is a function, $f(x)$, of the time $x$ (in hours). The hour hand is 10 cm long, and the middle of the clock face is placed at a height of 150 cm from the ground. Find the amplitude, midline and period of the function $f(x)$. Then, draw the graph of $f$ over one period and write the equation for $f$. 	
	\end{enumerate}	
	
	\par\medskip\hrule\medskip


\begin{mdframed}[style=exampledefault,frametitle={Looking Ahead to Next Week...}]
	{\underline{Definitions that you should be familiar with by the next class meeting:} }
	\begin{enumerate}[label= {  \arabic*:},labelindent=2em, style = standard,leftmargin=4pc, labelsep=*, noitemsep]
	\item Unit circle and terminal points on the circle;
	\item Periodic function;
	\item Sine and cosine functions;
	\item Amplitude, period and midline.
	\end{enumerate}

	
		{\underline{By the next class you should be able to::} }
		\begin{enumerate}[label= {  \arabic*:},labelindent=2em, style = standard,leftmargin=4pc, labelsep=*, noitemsep]
		\item Plot a number on the unit circle and identify its terminal point;
		\item Draw the graphs of the sine and cosine functions;
		\item Find the period of a periodic function, find its amplitude and write the equation of its midline;
		\item Find the domain and range of a periodic function, and write them in interval form;
		\item Use the trig wheel provided on teh next page to find sine and cosine of numbers.
		
		\end{enumerate}
			
\end{mdframed}	
%\end{enumerate}
\newpage

\par\medskip\hrule\medskip


\begin{tikzpicture}[scale=5.3,cap=round,>=latex]
% draw the coordinates
\draw[->] (-1.5cm,0cm) -- (1.5cm,0cm) node[right,fill=white] {$x$};
\draw[->] (0cm,-1.5cm) -- (0cm,1.5cm) node[above,fill=white] {$y$};

% draw the unit circle
\draw[thick] (0cm,0cm) circle(1cm);

\foreach \x in {0,30,...,360} {
	% lines from center to point
	\draw[gray] (0cm,0cm) -- (\x:1cm);
	% dots at each point
	\filldraw[black] (\x:1cm) circle(0.4pt);
	% draw each angle in degrees
	%\draw (\x:0.6cm) node[fill=white] {$\x^\circ$};
}

\foreach \x in {0,45,...,360} {
	% lines from center to point
	\draw[gray] (0cm,0cm) -- (\x:1cm);
	% dots at each point
	\filldraw[black] (\x:1cm) circle(0.4pt);
	% draw each angle in degrees
	%\draw (\x:0.6cm) node[fill=white] {$\x^\circ$};
}
% draw each angle in radians
\foreach \x/\xtext in {
	30/\frac{\pi}{6},
	45/\frac{\pi}{4},
	60/\frac{\pi}{3},
	90/\frac{\pi}{2},
	120/\frac{2\pi}{3},
	135/\frac{3\pi}{4},
	150/\frac{5\pi}{6},
	180/\pi,
	210/\frac{7\pi}{6},
	225/\frac{5\pi}{4},
	240/\frac{4\pi}{3},
	270/\frac{3\pi}{2},
	300/\frac{5\pi}{3},
	315/\frac{7\pi}{4},
	330/\frac{11\pi}{6},
	360/2\pi}
\draw (\x:0.85cm) node[fill=white] {$\xtext$};

\foreach \x/\xtext/\y in {
	% the coordinates for the first quadrant
	30/\frac{\sqrt{3}}{2}/\frac{1}{2},
	45/\frac{\sqrt{2}}{2}/\frac{\sqrt{2}}{2},
	60/\frac{1}{2}/\frac{\sqrt{3}}{2},
	% the coordinates for the second quadrant
	150/-\frac{\sqrt{3}}{2}/\frac{1}{2},
	135/-\frac{\sqrt{2}}{2}/\frac{\sqrt{2}}{2},
	120/-\frac{1}{2}/\frac{\sqrt{3}}{2},
	% the coordinates for the third quadrant
	210/-\frac{\sqrt{3}}{2}/-\frac{1}{2},
	225/-\frac{\sqrt{2}}{2}/-\frac{\sqrt{2}}{2},
	240/-\frac{1}{2}/-\frac{\sqrt{3}}{2},
	% the coordinates for the fourth quadrant
	330/\frac{\sqrt{3}}{2}/-\frac{1}{2},
	315/\frac{\sqrt{2}}{2}/-\frac{\sqrt{2}}{2},
	300/\frac{1}{2}/-\frac{\sqrt{3}}{2}}
\draw (\x:1.25cm) node[fill=white] {$\left(\xtext,\y\right)$};

% draw the horizontal and vertical coordinates
% the placement is better this way
\draw (-1.25cm,0cm) node[above=1pt] {$(-1,0)$}
(1.25cm,0cm)  node[above=1pt] {$(1,0)$}
(0cm,-1.25cm) node[fill=white] {$(0,-1)$}
(0cm,1.25cm)  node[fill=white] {$(0,1)$};
\end{tikzpicture}


\setlength\fboxrule{2pt}\setlength\fboxsep{2mm}
\fbox{This chart will be provided in the quizzes and in the final exam.} 

\end{document} 
              