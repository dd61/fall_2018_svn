\documentclass[11pt,answers]{exam}

\usepackage{etex}
\usepackage{amssymb,amsmath,multicol} %<-- InWorksheetExam1 i also have fancyhdr,

\usepackage[metapost]{mfpic}
\usepackage[pdftex]{graphicx}
\usepackage{tabu}

\usepackage{pst-plot}
\usepackage{pgfplots}
\pgfplotsset{compat=1.9}

\usepackage{tikz}
\usepackage{tkz-2d}
\usepackage{tkz-base}
\usetikzlibrary{calc}
\usetikzlibrary{arrows}

\usepackage{systeme}

\usepackage[inline]{enumitem}
\usepackage{refcount}%<-- non in WorksheetExam1

\usepackage{pstricks-add,pst-eucl}
\usepackage{systeme}
\usepackage{setspace}
\usepackage{multicol}


\usepackage[inline]{enumitem}   
\makeatletter
% This command ignores the optional argument for itemize and enumerate lists
\newcommand{\inlineitem}[1][]{%
\ifnum\enit@type=\tw@
    {\descriptionlabel{#1}}
  \hspace{\labelsep}%
\else
  \ifnum\enit@type=\z@
       \refstepcounter{\@listctr}\fi
    \quad\@itemlabel\hspace{\labelsep}%
\fi}
\makeatother


\def\f{x+1} \def\g{-x/3+2}  \def\h{-x+3}

\newcommand{\vasymptote}[2][]{
    \draw [densely dashed,#1] ({rel axis cs:0,0} -| {axis cs:#2,0}) -- ({rel axis cs:0,1} -| {axis cs:#2,0});
}

\addpoints
%\printanswers
\noprintanswers

\opengraphsfile{Q4a_Fa18}

\begin{document}
\extrawidth{-0.3in}
\pagestyle{headandfoot}

\setlength{\hoffset}{-.25in}

\extraheadheight{-.3in}
\runningheadrule
\firstpageheader{\bfseries {Precalculus}}{ \bfseries {Quiz 4 }}{\bfseries {10/2/18}} 

\begin{center}
	This quiz has \numquestions\ questions, for a total of \numpoints\
	points and \numbonuspoints\ bonus points.
\end{center}


\firstpagefooter{} %%&&CHANGED
                {}
                {%Points earned: \hbox to 0.5in{\hrulefill}
                % out of  \pointsonpage{\thepage} points
                }
                 
						

\vspace*{0.1cm}
\hbox to \textwidth { \scshape {Name:} \enspace\hrulefill}
\vspace{0.1cm}




\pointpoints{point}{points}

\begin{questions}


\addpoints

\question A tank holds 100 gallons of water, which drains from a leak at the bottom, causing the tank to drain in 40 minutes. The volume of water remaining in the tank after $t$ minutes is $\displaystyle V=f(t)=100\left (1-\frac{t}{40}\right )^2$.
\begin{parts}
	\part[2]  Find $\displaystyle f^{-1}(100)$. Show your work step by step.
	\fillwithdottedlines{2cm} 
	\part[1] What does $\displaystyle f^{-1}(19)$ represent? 
	\begin{choices}
		\choice $\displaystyle f^{-1}(19)$ represents the rate at which the tank is leaking when 19 gallons have drained from the tank.
		\choice $\displaystyle f^{-1}(19)$ represents the volume of water that has leaked out of the tank after 19 minutes.
		\choice $\displaystyle f^{-1}(19)$ represents the time that remains for the tank to be emptied when 19 gallons have drained from the tank.
		\choice $\displaystyle f^{-1}(19)$ represents the time that has elapsed when there are 19 gallons left in the tank. 
		\choice $\displaystyle f^{-1}(19)$ represents the water remaining in the tank 19 minutes after the tank started to leak.
	\end{choices}
\end{parts} 

\question Let $\displaystyle f(x)=\frac{1}{x}$ and $\displaystyle g(x)=\frac{1}{x-1}$.

\begin{parts}
	\part[2] Write the domain of $\displaystyle \frac{f}{g}$ in interval form. Show Your work step-by-step.
	\fillwithdottedlines{1cm}
		\part[2] Write the domain of $\displaystyle \frac{g}{f}$  in interval form. Show Your work step-by-step.
		\fillwithdottedlines{1cm}
	
	\end{parts}
\bonusquestion[1] The function shown below is not one-to-one.

\begin{minipage}{0.5\linewidth}
\begin{mfpic}[20]{-1}{6}{-1}{4}
	
	%\polyline{(0,-2), (4,1), (4,2), (5,3)}
	
	\polyline{(0,1), (2,0)} 
	
	\polyline{(2,0), (4,2)}
	
	\point[5pt]{(0,1), (4,2)}
	%\tlabel[cc](4,2.5){\scriptsize $(4,2)$}
	\axes
	
	\xmarks{1,2,3,4}
	
	\ymarks{-1,1,2,3,4,}
	
	\tlpointsep{4pt}
	
	\axislabels {x}{{\tiny $1$} 1, {\tiny $2$} 2, {\tiny $3$} 3, {\tiny $4$} 4, {\tiny $5$} 5}
	
	\axislabels {y}{{\tiny $1$} 1,{\tiny $2$} 2, {\tiny $3$} 3,   {\tiny $-1$} -1}
	
	% Grid
	%\drawcolor[gray]{0.005}
	%\gridlines{1, 1}
	\drawcolor[gray]{0.95} 
	\grid{1,1}
	
\end{mfpic}

\end{minipage}
\begin{minipage}{0.5\linewidth}
	Restrict its domain so that the resulting function is one-to-one. Write your answer in interval form. (Note: there are many possible correct answers).
	\fillwithdottedlines{2cm}
\end{minipage}

\end{questions}

\end{document}                 