\documentclass[12pt,dvipsnames]{article}
\usepackage[margin=0.4in,footskip=0.1in]{geometry}
\usepackage{etex}
\usepackage{amssymb,amsmath,multicol} %<-- InWorksheetExam1 i also have fancyhdr,
\usepackage{hyperref}
\usepackage[metapost]{mfpic}
\usepackage[pdftex]{graphicx}
\usepackage{csquotes}
\usepackage{pst-plot}
\usepackage{pgfplots}
\pgfplotsset{compat=1.9}

\usepackage{tikz}
\usepackage{tkz-2d}
\usepackage{tkz-base}
\usetikzlibrary{calc}
\usepackage{color}
\usepackage[inline]{enumitem}
\usepackage{refcount}%<-- non in WorksheetExam1

%\usepackage[linewidth=1pt]{mdframed}


\usepackage[framemethod=TikZ]{mdframed}
\newcommand{\mdfLABEL}[1]{\node[text width=2em,align=right,anchor=north east,%
	outer sep=0pt,inner sep=0pt] at ($ (O|-P)
	-(\the\mdflength{innerleftmargin},0)
	-0.5*(\the\mdflength{middlelinewidth},0)
	- (0,\the\mdflength{innertopmargin})
	+ (0,0.5pt)
	$) {$\Box$};} %in the original code, $F$ is replaced with #1

\mdfdefinestyle{testframe}{topline=false,rightline=false,bottomline=false,%
	innerleftmargin=1em,linecolor=white,rightmargin=2em,skipbelow=1em,%
	tikzsetting={draw=black,line width=.5pt,dashed,dash pattern= on 1pt off 3pt},%
	firstextra={\mdfLABEL{(i)}},%
	singleextra={\mdfLABEL{(i)}},%
	secondextra={\mdfLABEL{$\phantom{.}$}},%
	middleextra={\mdfLABEL{$\phantom{.}$}},%
}



\usepackage{caption}
\usetikzlibrary{calc,fit,intersections,shapes,calc}
\usetikzlibrary{backgrounds}
\usepackage{systeme}
\usepackage{multicol}

\newcolumntype{?}{!{\vrule width 1pt}}


%%These three lines are for the typewriter font. Comment them out if I don't want the font.
%%%%%%\renewcommand*\ttdefault{lcmtt}
%%%%%%\renewcommand*\familydefault{\ttdefault} %% Only if the base font of the document is to be typewriter style
%%%%%%\usepackage[T1]{fontenc}
%%%%%%

\usepackage{tabularx, booktabs}


\newenvironment{myitemize}
{ \begin{itemize}
		\setlength{\itemsep}{10pt}
		\setlength{\parskip}{10pt}
		\setlength{\parsep}{10pt}     }
	{ \end{itemize}   
	
} 

\usepackage{setspace}

\font\maxi=cminch scaled 100
\usepackage{tgadventor}
%\renewcommand*\familydefault{\sfdefault} %% Only if the base font of the document is to be sans serif
\usepackage[T1]{fontenc}
\newcommand*{\myfont}{\fontfamily{\sfdefault}\selectfont}
\usepackage{pacioli}
\usepackage[OT1]{fontenc}
\usepackage{systeme}
%\usepackage{ulem}

%\usepackage{spalign}


%\usepackage{AlegreyaSans} %% Option 'black' gives heavier bold face
%% The 'sfdefault' option to make the base font sans serif
%\renewcommand*\oldstylenums[1]{{\AlegreyaSansOsF #1}}

\newcommand*\circled[1]{\tikz[baseline=(char.base)]{%
		\node[shape=circle,fill=blue!20,draw,inner sep=2pt] (char) {#1};}}

\usepackage{lastpage}
\usepackage{fancyhdr}
\pagestyle{fancy} 

\rfoot{{\small{Page \thepage\ of \pageref{LastPage}}}}
\cfoot{}
\renewcommand{\baselinestretch}{1.50}\normalsize




\makeatletter
\newenvironment{enumeratecount}[1]
{\def\thisenumeratecountlabel{#1}\enumerate}
{\edef\@currentlabel{\number\value{\@enumctr}}%
	\label{\thisenumeratecountlabel}\endenumerate}
\makeatother

\usepackage{refcount}
\newcommand{\addphrase}[3]{% #1 = label, #2 = text if number >1, #3 = text if number =1
	\ifnum\getrefnumber{#1}>1
	#2%
	\else
	#3%
	\fi}


\makeatletter
% This command ignores the optional argument for itemize and enumerate lists
\newcommand{\inlineitem}[1][]{%
	\ifnum\enit@type=\tw@
	{\descriptionlabel{#1}}
	\hspace{\labelsep}%
	\else
	\ifnum\enit@type=\z@
	\refstepcounter{\@listctr}\fi
	\quad\@itemlabel\hspace{\labelsep}%
	\fi}
\makeatother

\newcommand*\circledA[1]{\tikz[baseline=(char.base)]{%
		\node[shape=circle,fill=green!20,draw,inner sep=2pt] (char) {#1};}}

\opengraphsfile{Wk_14_DQ_Fa18}

\begin{document}
\thispagestyle{empty}

%	\thispagestyle{empty}
	\begin{center}
		{\large{Week 14}}
	\end{center}



	{\bfseries{Definitions to memorize in preparation of the final exam:}} 

\begin{description}[topsep=0pt,itemsep=-2ex,partopsep=0ex,parsep=1ex]
\item[From Weeks 1-14] Linear equations and inequalities, feasible region and objective function, linear programming algorithm,  function, domain, range, interval form, transformation, operations on functions, one-to-one function, inverse, polynomial, degree, leading coefficient, end behavior, zero, multiplicity, $x\to \infty$, $x\to -\infty$, rational function, arrow notation ($x\to a^{-}, x\to a^{+}, x\to \infty, x\to -\infty$), vertical asymptote, horizontal asymptote, slant asymptote, exponential function, graph of an exponential function, compound interest, annual percentage yield, $e$, natural exponential function, continuous compounding, $\log_a x$, $\log x$, $\ln x$, doubling time, half life, relative growth rate, unit circle, terminal point, reference point, periodic function, sine, cosine, amplitude, period, tangent, cotangent, phase shift, inverse sine, inverse cosine, $\arcsin$, $\arccos$.
\end{description}
\smallskip	


		
%\end{multicols}
{\bfseries{Bring to class:} } A paper notebook with your annotations of the reading and your work on the questions listed below; a pen and/or a sharpened pencil and an eraser.

{\bfseries{Laptops/Phones Policy:}}  No devices in class, unless the assignment requires it.

{\bfseries{Audio-Recording:}} I will be calling people (by name) from the class roster to go over the discussion questions: to ensure everyone's privacy, please do not audio-record the class.



% References for the above: 
% 1) https://www.math.utah.edu/~gustafso/s2016/2270/published-projects-2016/williamsBarrett/williamsBarrett-Fast-Fourier-Transform-Predicting-Financial-Securities-Prices.pdf

%2) http://peterelsea.com/Maxtuts_msp/FourierNotes_2010.pdf Good explanation of basic trig functions

\begin{center}

{\large{\bfseries{Discussion Questions for Week 14} }}
\end{center}
	%\begin{enumerate}[label=\protect\circled{\arabic*}]
		\renewcommand{\labelenumi}{(\arabic{enumi})}
		%%%%%%%%%%%%%%%%%%%
		%%%%% https://www.illustrativemathematics.org/HSF-LE.A
		%%%%% GREAT REFERENCE
		%%%%%%%%%%%%%%%%%%%



\begin{enumerate}[label= \protect\circled{\arabic*}]
%%%
%% 1
\item (Week 1, problem 5, pg 779) A furniture manufacturer makes wooden tables and chairs. The production process involves two basic types of labor: carpentry and finishing. A table requires 2 hours of carpentry and 1 hour of finishing, and a chair requires 3 hours of carpentry and $\displaystyle \frac{1}{2}$ hour of finishing.  The manufacturer's employees can supply a maximum of 108 hours of carpentry work and 20 hours of finishing work per day.The profit is \$35 per table and \$20 per chair. How many chairs and tables should be made each day to maximize profit? 


%%%
% 2

\item (Week 2) Consider the graph of the function $f(x)$:

\begin{minipage}{0.5\linewidth}
	\begin{center}
		\begin{mfpic}[20]{-5}{5}{-2}{4}
			
			\polyline{(-1,-1), (0,2)} 
			
			\polyline{(0,3), (1,3)}
			\polyline{(1,2),(4,0)}
			
			\point[5pt]{(-1,-1), (0,3), (1,2), (4,0)}
			\circle{(0, 2),0.20}
			\circle{(1, 3),0.20}
			
			\axes
			
			\xmarks{1,2,3,4,5}
			
			\ymarks{-2,-1,1,2,3,4,}
			
			\tlpointsep{4pt}
			\tlabel[cc](5,-0.8){\scriptsize $x$}
			\tlabel[cc](0.5,4){\scriptsize $y$}
			\tcaption{$y = f(x)$.}
			
			\axislabels {x}{{\tiny $1$} 1, {\tiny $2$} 2, {\tiny $3$} 3, {\tiny $4$} 4, {\tiny $5$} 5,  {\tiny $-1$} -1, {\tiny $-2$} -2}
			
			\axislabels {y}{{\tiny $1$} 1,{\tiny $2$} 2, {\tiny $3$} 3, {\tiny $4$} 4,  {\tiny $-1$} -1, {\tiny $-2$} -2}
			
			% Grid
			
			%\drawcolor[gray]{0.25}
			%\gridlines{1, 1}
			\drawcolor{1.05} 
			\grid{1,1}
			
		\end{mfpic}
		
	\end{center}
\end{minipage}
\begin{minipage}{0.5\linewidth}
	\begin{enumerate}
		\item Write the domain and range of $f(x)$ in interval form.
		\item  Write the equation for $f(x)$.
		\item Find all the $x$ values for which $f(x)\geq 1$ and write Your answer in interval form.
	\end{enumerate}
	
\end{minipage}

%%
% 3

\item (Week 3) The graph of a function $f(x)$ is shown below.



\begin{center}
	
	
	
	\begin{mfpic}[20]{-1}{6}{-2}{4}
		
		%\polyline{(0,-2), (4,1), (4,2), (5,3)}
		
		\polyline{(1,0), (5,3)} 
		
		%\polyline{(2,2), (5,2)}
		
		\point[5pt]{(1,0),  (5,3)}
		%\circle{(2,2),0.15}
		%\circle{(5,2),0.15}
		
		
		%\tlabel[cc](-1,1){\scriptsize $(0,1)$}
		
		%\tlabel[cc](2,3.5){\scriptsize $(2,3)$}
		
		%\tlabel[cc](4.5,2.5){\scriptsize $(4,2)$}
		
		%\tlabel[cc](5,-0.5){\scriptsize $(5,0)$}
		
		%\tlabel[cc](6,-0.5){\scriptsize $x$}
		
		%\tlabel[cc](0.5,6){\scriptsize $y$}
		
		
		%\tcaption{\scriptsize $y=f(x)$}
		
		\axes
		
		\xmarks{1,2,3,4,5}
		
		\ymarks{-2,-1,1,2,3}
		
		\tlpointsep{4pt}
		
		\axislabels {x}{{\tiny $1$} 1, {\tiny $2$} 2, {\tiny $3$} 3, {\tiny $4$} 4, {\tiny $5$} 5}
		
		\axislabels {y}{{\tiny $1$} 1,{\tiny $2$} 2, {\tiny $3$} 3,  {\tiny $-1$} -1, {\tiny $-2$} -2}
		
		% Grid
		%\drawcolor[gray]{0.25}
		%\gridlines{1, 1}
		\drawcolor[gray]{0.95} 
		\grid{1,1}
		
	\end{mfpic}
	
\end{center}

\begin{enumerate}
	\item  Fill out the table: 
	\bigskip
	
	
	
	\begin{minipage}{\linewidth}
		\centering
		
		{\setlength{\tabcolsep}{1.9em}  
			{\renewcommand{\arraystretch}{2}%
				\begin{tabular}{|l|l|l|l|l|}
					\hline
					$x$    & $1$ & $2$ & $3$ & $4$   \\ \hline
					$f(x-1) $ &      &     &   & \\ \hline
				\end{tabular}}} \quad
			\end{minipage}
			
			\item Fill out the table: 
			\bigskip
			
			
			
			\begin{minipage}{\linewidth}
				\centering
				
				{\setlength{\tabcolsep}{1.9em}  
					{\renewcommand{\arraystretch}{2}%
						\begin{tabular}{|l|l|l|l|l|}
							\hline
							$x$    & $1$ & $2$ & $3$ & $4$   \\ \hline
							$f(2x) $ &      &     &   & \\ \hline
						\end{tabular}}} \quad
					\end{minipage}
				\end{enumerate}
				
				
				\item (Week 3) The indicated transformations are applied to the  graph of $\displaystyle f(x) = x^2$ (in the given order):
				\smallskip 
				
				{\emph{shift 1 unit to the right, then shift 1 unit down, then reflect in the $x$-axis.}}
				\smallskip
				
				
				Write the equation for the final transformed graph. 

%%
%4

\item (Week 4) Two functions $f$ and $g$ are shown below.

\begin{minipage}{0.5\linewidth}
	\begin{center}
		
		\begin{mfpic}[20]{-3}{6}{-2}{5}
			
			\polyline{(1,0), (5,4)} 
			%\polyline{(2,0), (4,2)}
			\point[5pt]{(1,0), (5,4)}
			\tcaption{\scriptsize $y=f(x)$}
			\axes
			\xmarks{-2,-1,1,2,3,4,5}
			\ymarks{-2,-1,1,2,3,4,}
			\tlpointsep{4pt}
			\axislabels {x}{{\tiny $-2$} -2,{\tiny $-1$} -1,{\tiny $1$} 1, {\tiny $2$} 2, {\tiny $3$} 3, {\tiny $4$} 4, {\tiny $5$} 5}
			\axislabels {y}{{\tiny $1$} 1,{\tiny $2$} 2, {\tiny $3$} 3, {\tiny $4$} 4,  {\tiny $-1$} -1, {\tiny $-2$} -2}
			\drawcolor[gray]{0.75} 
			\grid{1,1}
		\end{mfpic}
	\end{center}
\end{minipage}
\begin{minipage}{0.5\linewidth}
	\begin{center}
		
		\begin{mfpic}[20]{-1}{6}{-2}{5}
			
			%\polyline{(0,-2), (4,1), (4,2), (5,3)}
			
			\polyline{(0,4), (4,0)} 
			
			%\polyline{(3,3), (5,4)}
			
			\point[5pt]{(0,4), (4,0)}
			%\circle{(3, 3),0.15}
			\tcaption{\scriptsize $y=g(x)$}
			\axes
			
			\xmarks{1,2,3,4,5}
			
			\ymarks{-2,-1,1,2,3,4,}
			
			\tlpointsep{4pt}
			
			\axislabels {x}{{\tiny $1$} 1, {\tiny $2$} 2, {\tiny $3$} 3, {\tiny $4$} 4, {\tiny $5$} 5}
			
			\axislabels {y}{{\tiny $1$} 1,{\tiny $2$} 2, {\tiny $3$} 3, {\tiny $4$} 4,  {\tiny $-1$} -1, {\tiny $-2$} -2}
			
			\drawcolor[gray]{0.75} 
			\grid{1,1}
			
		\end{mfpic}
	\end{center}
\end{minipage}

\begin{enumerate}
	\item \label{ettob1}  Fill out the table:
	
	
	\begin{minipage}{\linewidth}
		\centering
		\captionof{table}{$f(g(x))$} %\label{tab:title} 
		\begin{tabularx}{0.8\textwidth}{|X|X|X|X|X|X|X|X|}
			\hline
			\multicolumn{2}{|c|}{$x$}         & $0$ & $1$ & $2$ & $3$ & $4$ & 5  \\ \hline
			\multicolumn{2}{|c|}{$f(g(x))$}   & & &     &     &    &         \\ \hline
		\end{tabularx}
	\end{minipage}
	\item What is the domain of $\displaystyle f(g(x))$?
	\item Redo part \ref{ettob1} replacing $f(g(x))$ with $g(f(x))$.
	\item What is the domain of $\displaystyle g(f(x))$?
	\item Find formulas for $\displaystyle f^{-1}$ and $g^{-1}$. Find the domain and range of each function.
\end{enumerate}

%%
% 5

\item (Week 5) List the function transformations of $\displaystyle f(x) = x^2$ that result in $\displaystyle g(x)=2x^2+4x-1$. 
\item (Week 5) Consider the polynomial function: $\displaystyle f(x)=x^^4(x-3)(x+1)^2$. Find: 
\begin{enumerate*}[label=\roman*),itemjoin={;\quad}]
	\item the domain; 
	\item the range;
	\item any $x$-intercepts;
	\item the $y$-intercept, if it exists;
	\item  the end  behavior;
	\item the largest number of extrema (number of maxima + number of minima) that the polynomial function may have;
	\item the shape of the graph near each $x$-intercept.
\end{enumerate*}
Sketch the graph of $f(x)$ using the information you have gathered about the polynomial.

%%
% 6

\item (Week 6) Write an equation for a rational function $r(x)$ with all of the following properties:
\begin{itemize}
	\item The graph has exactly one vertical asymptote, $x=3$;
	\item The graph has an $x$ intercept at $(1,0)$, and the graph bounces off the $x$-axis at $(1,0)$;
	\item The $y$ intercept is $(0,4)$.
	\item The horizontal asymptote is $y=1$.
\end{itemize}
\item (Week 6, from an old exam) This problem is about the function $\displaystyle r(x)=\frac{(x-1)^2(x+1)(x+2)}{(x-2)^2(x+0.5)(x-0.5)}$.
\begin{enumerate}
	\item Write the domain in interval form).
	\item Find the $x$ intercept(s) and $y$ intercept, if they exist.
	\item 	Write the equation(s) of  the vertical asymptote(s). 
	\item Write the equation of the horizontal asymptote and explain your reasoning.
	\item (bonus question) 	Draw a possible graph for $r(x)$. In order to receive any credit for this question, you must include {\bfseries{all}} of the following items:
	\begin{enumerate}
		\item Your graph includes units both on the $x$ and the $y$ axes.
		\item You have explained how you know the behavior of the function near the vertical asymptote(s). (Do the tails go in the same direction or in opposite direction? How do you know?)
		\item If the function has $x$ intercept(s), you have stated whether the graph goes across the $x$ axis or bounces off, and how you can see this from the equation.
	\end{enumerate}
		\end{enumerate}

%%
%%12

%%
% 8
\item (Week 8) An investment grows by 1\% per year for 10 years. By what percent does it increase over a 10-year period?

\item (Week 8) A grandmother plans to invest in a college savings plan for a period of 18 years: the plan will pay an interest of 2\% compounded semiannually, and the grandmother would like the plan to grow to \$40,000 after 18 years. How much will she need to invest?

\item (Week 9) Evaluate $\log(\log(10))$ without a calculator.

\item (Week 9) Write the domain and range of $f(x)=\log(x^2-1)$ in interval form.

\item (Week 9) Simplify $\displaystyle \log(x^2\sqrt{y})$ as much as possible.



%%
%10

\item (Week 10)  If \$1000 is invested in an account paying an interest of 0.5\% compounded monthly, how long will it take for the account to grow to \$1100?

\item (Week 10) Find the exact value(s) of the solution(s) of the equation $\displaystyle 17e^{0.02x}=18e^{0.03x}$.

\item (Week 10) A hot bowl of soup is served at a dinner party. It starts to cool according to Newton's Law of Cooling so that its temperature at  time $t$ is given by
$\displaystyle T(t) = 60 + 140e^{-0.05t}$
where $t$ is measured in minutes and $T$ is measured in ${}^{0}$F.
\begin{enumerate}
	\item Find the initial temperature of the soup. 

	\item Find the time it takes for the temperature of the soup to be $100^0$F. [Note: your answer will be a fraction.]
	\end{enumerate}
	
	\item (Week 10) 	 If \$10,000 is deposited in an account that pays interest at an annual rate of 0.3\% compounded continuously, how long does it take for the account to double?
	
%%
%11

\item (Week 11) Find the terminal point of $\displaystyle t=-\frac{41\pi}{3}$ on the unit circle.

\item (Week 11) 	List the transformations of $h(t)=\cos t$ that result in each of the functions listed below. Find the period, domain, range, amplitude and midline of each of these functions.

\begin{multicols}{3}
	\begin{enumerate}[label=\fbox{\arabic*}]
		\item $\displaystyle y=-\cos t$;
		\item $\displaystyle y=\cos(-t)$;
		\item $\displaystyle y=\cos(2t)$;
		\item $\displaystyle y=\cos \left (\frac{t}{2}\right )$;
		\item $\displaystyle y=1+\cos t$;
		%\item $\displaystyle y=1-\sin t$;
		%\item $\displaystyle y=\sin(2t)$;
		\item $\displaystyle y=\cos\left(\frac{\pi}{2}t\right)$;
		%\item $\displaystyle y=2\sin t$;
		\item $\displaystyle y=\frac{\cos t}{2}$;
		\item $\displaystyle y=\cos\left(t+\frac{\pi}{2}\right)$;
		%\item $\displaystyle y=\sin\left(t-\frac{\pi}{2}\right)$.
	\end{enumerate}
\end{multicols}	

\item (Week 12) The height (in cm) of the tip of the hour hand on a vertical clock face is a function, $f(x)$, of the time $x$ (in hours). The hour hand is 10 cm long, and the middle of the clock face is placed at a height of 150 cm from the ground. Find the amplitude, midline and period of the function $f(x)$. Then, draw the graph of $f$ over one period and write the equation for $f$. 

	\item (Week 12) Part of the graphs of two sinusoidal functions are shown below. For each function, write an equation that represents the graph in the form
	$\displaystyle y=A\sin(B(t-h))+k$. 
		\begin{multicols}{2}
			\begin{enumerate}
						\item Function $f(x)$:
											
						\begin{tikzpicture}
						\begin{axis}[
						%minor tick num=3,
						axis y line=center,
						axis x line=middle,
						%enlarge x limits=0.15,
						enlarge y limits=0.15,
						every axis y label/.style={at={(current axis.above origin)},anchor=north east},
						xlabel=$x$,ylabel=$y$,xtick={-6,-5,-4,-3,-2,-1,0,1,2,3,4,5,6}, xticklabels={-6,-5,-4,-3,-2,-1,0,1,2,3,4,5,6},ytick={-3,-2,-1,0,1},yticklabels={-3,-2,-1,0,1},grid=both, minor tick num=1
						]
						\addplot[smooth,blue,mark=none,
						domain=-6:6,samples=40] 
						{cos(deg(x*pi/5+2*pi/5))};
						\end{axis}
						\end{tikzpicture}
						
						\item Function $g(x)$:
						
						\begin{tikzpicture}
						\begin{axis}[
						%minor tick num=3,
						axis y line=center,
						axis x line=middle,
						%enlarge x limits=0.15,
						enlarge y limits=0.15,
						every axis y label/.style={at={(current axis.above origin)},anchor=north east},
						xlabel=$x$,ylabel=$y$,xtick={-6,-5,-4,-3,-2,-1,0,1,2,3,4,5,6}, xticklabels={-6,-5,-4,-3,-2,-1,0,1,2,3,4,5,6},ytick={-3,-2,-1,0,1,2},yticklabels={-3,-2,-1,0,1,2},grid=both, minor tick num=1
						]
						\addplot[smooth,blue,mark=none,
						domain=-4:4,samples=40] 
						{-2*cos(deg(2*x*pi/3))};
						\end{axis}
						\end{tikzpicture}			
						
					\end{enumerate}
				\end{multicols}
				

%%%
%% 13

	\item (Week 13)  A Ferris wheel with 59 capsules has a diameter of 110 meters and its lowest point (at the 6 o'clock position) is 5 meters above the ground. One full rotation of the wheel takes 24 minutes, and the wheel rotates counterclockwise at a constant speed. The capsules are labeled from 1 to 59. We start observing when capsule 1 is at the lowest point on the wheel. If $H(t)$ is the height of capsule 1 $t$ minutes after we start the observation, when is capsule 1 going to be at a height of 70 meters from the ground during the first rotation? 
	


\item (Week 13) A person's body temperature fluctuates during the day following  Circadian rhythms. The lowest temperature, 36.6 degrees centigrades, occurs at 4 am, and the highest temperature, 37.4 degrees centigrades, occurs at 4 pm. Write a formula for a sinusoidal function that models the person's body temperature throughout the day, assuming that $t=0$ at midnight. Then find the times when the body temperature is at least 37 degrees centigrades.


	\end{enumerate}
	
	\par\medskip\hrule\medskip



%\end{enumerate}
\newpage

\setlength\fboxrule{2pt}\setlength\fboxsep{2mm}
\fbox{This chart will be provided in the quizzes and in the final exam.} 

\par\medskip\hrule\medskip


\begin{tikzpicture}[scale=5.3,cap=round,>=latex]
% draw the coordinates
\draw[->] (-1.5cm,0cm) -- (1.5cm,0cm) node[right,fill=white] {$x$};
\draw[->] (0cm,-1.5cm) -- (0cm,1.5cm) node[above,fill=white] {$y$};

% draw the unit circle
\draw[thick] (0cm,0cm) circle(1cm);

\foreach \x in {0,30,...,360} {
	% lines from center to point
	\draw[gray] (0cm,0cm) -- (\x:1cm);
	% dots at each point
	\filldraw[black] (\x:1cm) circle(0.4pt);
	% draw each angle in degrees
	%\draw (\x:0.6cm) node[fill=white] {$\x^\circ$};
}

\foreach \x in {0,45,...,360} {
	% lines from center to point
	\draw[gray] (0cm,0cm) -- (\x:1cm);
	% dots at each point
	\filldraw[black] (\x:1cm) circle(0.4pt);
	% draw each angle in degrees
	%\draw (\x:0.6cm) node[fill=white] {$\x^\circ$};
}
% draw each angle in radians
\foreach \x/\xtext in {
	30/\frac{\pi}{6},
	45/\frac{\pi}{4},
	60/\frac{\pi}{3},
	90/\frac{\pi}{2},
	120/\frac{2\pi}{3},
	135/\frac{3\pi}{4},
	150/\frac{5\pi}{6},
	180/\pi,
	210/\frac{7\pi}{6},
	225/\frac{5\pi}{4},
	240/\frac{4\pi}{3},
	270/\frac{3\pi}{2},
	300/\frac{5\pi}{3},
	315/\frac{7\pi}{4},
	330/\frac{11\pi}{6},
	360/2\pi}
\draw (\x:0.85cm) node[fill=white] {$\xtext$};

\foreach \x/\xtext/\y in {
	% the coordinates for the first quadrant
	30/\frac{\sqrt{3}}{2}/\frac{1}{2},
	45/\frac{\sqrt{2}}{2}/\frac{\sqrt{2}}{2},
	60/\frac{1}{2}/\frac{\sqrt{3}}{2},
	% the coordinates for the second quadrant
	150/-\frac{\sqrt{3}}{2}/\frac{1}{2},
	135/-\frac{\sqrt{2}}{2}/\frac{\sqrt{2}}{2},
	120/-\frac{1}{2}/\frac{\sqrt{3}}{2},
	% the coordinates for the third quadrant
	210/-\frac{\sqrt{3}}{2}/-\frac{1}{2},
	225/-\frac{\sqrt{2}}{2}/-\frac{\sqrt{2}}{2},
	240/-\frac{1}{2}/-\frac{\sqrt{3}}{2},
	% the coordinates for the fourth quadrant
	330/\frac{\sqrt{3}}{2}/-\frac{1}{2},
	315/\frac{\sqrt{2}}{2}/-\frac{\sqrt{2}}{2},
	300/\frac{1}{2}/-\frac{\sqrt{3}}{2}}
\draw (\x:1.25cm) node[fill=white] {$\left(\xtext,\y\right)$};

% draw the horizontal and vertical coordinates
% the placement is better this way
\draw (-1.25cm,0cm) node[above=1pt] {$(-1,0)$}
(1.25cm,0cm)  node[above=1pt] {$(1,0)$}
(0cm,-1.25cm) node[fill=white] {$(0,-1)$}
(0cm,1.25cm)  node[fill=white] {$(0,1)$};
\end{tikzpicture}





\end{document} 
              