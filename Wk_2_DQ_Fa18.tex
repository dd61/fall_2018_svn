\documentclass[12pt,dvipsnames]{article}
\usepackage[margin=0.4in,footskip=0.1in]{geometry}
\usepackage{etex}
\usepackage{amssymb,amsmath,multicol} %<-- InWorksheetExam1 i also have fancyhdr,
\usepackage{hyperref}
\usepackage[metapost]{mfpic}
\usepackage[pdftex]{graphicx}
\usepackage{csquotes}
\usepackage{pst-plot}
\usepackage{pgfplots}
\pgfplotsset{compat=1.9}

\usepackage{tikz}
\usepackage{tkz-2d}
\usepackage{tkz-base}
\usetikzlibrary{calc}
\usepackage{color}
\usepackage[inline]{enumitem}
\usepackage{refcount}%<-- non in WorksheetExam1

\usepackage[linewidth=1pt]{mdframed}

\usepackage{caption}
\usetikzlibrary{calc,fit,intersections,shapes,calc}
\usetikzlibrary{backgrounds}
\usepackage{systeme}


%%These three lines are for the typewriter font. Comment them out if I don't want the font.
%%%%%%\renewcommand*\ttdefault{lcmtt}
%%%%%%\renewcommand*\familydefault{\ttdefault} %% Only if the base font of the document is to be typewriter style
%%%%%%\usepackage[T1]{fontenc}
%%%%%%

\usepackage{tabularx, booktabs}


\newenvironment{myitemize}
{ \begin{itemize}
		\setlength{\itemsep}{10pt}
		\setlength{\parskip}{10pt}
		\setlength{\parsep}{10pt}     }
	{ \end{itemize}   
	
} 

\usepackage{setspace}

\font\maxi=cminch scaled 100
\usepackage{tgadventor}
%\renewcommand*\familydefault{\sfdefault} %% Only if the base font of the document is to be sans serif
\usepackage[T1]{fontenc}
\newcommand*{\myfont}{\fontfamily{\sfdefault}\selectfont}
\usepackage{pacioli}
\usepackage[OT1]{fontenc}
\usepackage{systeme}


%\usepackage{spalign}


%\usepackage{AlegreyaSans} %% Option 'black' gives heavier bold face
%% The 'sfdefault' option to make the base font sans serif
%\renewcommand*\oldstylenums[1]{{\AlegreyaSansOsF #1}}

\newcommand*\circled[1]{\tikz[baseline=(char.base)]{%
		\node[shape=circle,fill=blue!20,draw,inner sep=2pt] (char) {#1};}}

\usepackage{lastpage}
\usepackage{fancyhdr}
\pagestyle{fancy} 

\rfoot{{\small{Page \thepage\ of \pageref{LastPage}}}}
\cfoot{}
\renewcommand{\baselinestretch}{1.50}\normalsize



\opengraphsfile{Wk_2_DQ_Fa18}

\begin{document}
\thispagestyle{empty}

%	\thispagestyle{empty}
	\begin{center}
		{\large{Week 2}}
	\end{center}

{\bfseries{Textbook sections to read and annotate before class:}}  2.1, 2.2, 2.3 (skip examples 3 and 4), 2.5. If needed, please review interval notation (pg 7.)
\smallskip

	{\bfseries{Definitions to memorize before class:}} expression; equation; identity; function; input; output; domain; range.
\smallskip	
	
{\bfseries{Skills to review before class:} }
	\begin{enumerate} 
		\item Graphing intervals and writing them in interval form;
		\item Evaluating expressions;
                     \item Solving quadratic equations by factoring and with the quadratic formula;
		\item Finding the values of $x$ for which an algebraic fraction makes sense;
		\item Explaining the difference between a relation and a function (see below).
\end{enumerate}
	
		
{\bfseries{Bring to class:} } A paper notebook with your annotations of the reading and your work on the questions listed below; a pen and/or a sharpened pencil and an eraser.

{\bfseries{Laptops/Phones Policy:}}  No devices in class, unless the assignment requires it.

{\bfseries{Audio-Recording:}} I will be calling people (by name) from the class roster to go over the discussion questions: to ensure everyone's privacy, please do not audio-record the class.


\begin{center}

{\large{\bfseries{Reading and Discussion Questions for Week 2} }}
\end{center}
A relation is any collection of ordered pairs. If we denote the ordered pairs in a relation by $(x,y)$ then the set of $x$-values (or inputs) is the domain and the set of $y$-values (or outputs) is the range. With this terminology a function is a relation where for each $x$-value there is exactly one $y$-value (in this case, we say $y$ is a function of $x$.).

Your book gives this mathematical definition of the word function: a function is a rule that assigns to each element of a set A (let's say a generic element is called $x$) a unique element (which we call $f(x)$, and read ``f of x") of a set B (the set B may coincide with A, but it doesn't have to).

The two key words here are {\textcolor{blue}{\underline{each} }} and {\textcolor{blue}{\underline{unique}}}.

For example, the temperature in degrees Fahrenheit is a function in of the temperature in degrees Celsius, because each temperature in 
Celsius corresponds to one temperature in Fahrenheit (the converse is true as well, that is, the temperature in degrees Celsius is a 
function of the temperature in degrees Fahrenheit).

In plain English, the word \enquote{function} has a broader and less specific meaning. In the questions below, and in the rest of the course, when we use the word \enquote{function} we mean the mathematical definition of the word. 

\begin{enumerate}[label=\protect\circled{\arabic*}]
	\item At the beginning of this week's reading (pg 148), your textbook describes 8 functions verbally (\enquote{height is a function of age}, \enquote{temperature is a function of date}, etc.). Explain in your own words why the statement \enquote{age is a function of height} is false. 
\item Let $h(t)$ be the height above ground, in feet, of a rocket $t$ seconds after launching.
Explain the meaning of $h(1)=200$ in practical terms.
\item A function $f(x)$ is given by the table shown below:


\begin{tabular}{|l|l|l|l|l|l|l|l|l|l|l|}
\hline
$x$    & 0  & 1  & 2 & 3  & 4  & 5 & 6  & 7  & 8  & 9  \\ \hline
$f(x)$& 74 & 28 & 1 & 53 & 56 & 3 & 36 & 45 & 14 & 47 \\ \hline
\end{tabular}


First, find $f(3)$. Then, solve the equation $f(x)=36$.

	\item In example 1c on pg 150, you have read about the domain and range of the function $f$ defined by the formula $\displaystyle f(x)=x^2+4$.  In the next two questions, we will change the formula for $f$ a little bit.
	\begin{enumerate}
		\item What is the domain of $\displaystyle f(x)= \frac{1}{x^2}+4$? Review interval notation (pg 7 of the textbook) and write the domain in interval notation.
		\item What is the domain of $\displaystyle f(x)= \frac{1}{x^2-4}$? Write the domain in interval notation.
		\item What is the domain of $\displaystyle f(x)= \sqrt{x}+4$? Write the domain in interval notation.
\end{enumerate}


\item Explain in Your own words what we mean by \enquote{piecewise defined function}. Then, describe the domain and range of the  piecewise defined function $C(x)$ described iIn example 3 on pg 151 of  your textbook.
\item Graph the piecewise defined function from example 3 on pg 151. Refer to example 4 on pg 162 to include solid dots/open dots as needed.

\item Find an algebraic and a graphical representation for the function described verbally as follows:  \enquote{To evaluate F(x), subtract 1 from the input and divide the result by 2.}

\item What is the vertical line test, and what is it used for? 

\item \label{vertical}  Consider the graph:
	
	\begin{minipage}{0.5\linewidth}  
		\begin{center}
			
			
			
			\begin{mfpic}[20]{-1}{6}{-2}{5}
				
				
				\polyline{(0,2), (2,2)} 
				
				\polyline{(2,1), (5,1)}
				
				\point[5pt]{(0,2),  (2,2), (2,1),(5,3)}
				\circle{(5,1),0.15}
				%\circle{(2,2),0.15}
				\axes
				
				\xmarks{1,2,3,4,5}
				
				\ymarks{-2,-1,1,2,3,4,}
				
				\tlpointsep{4pt}
				
				\axislabels {x}{{\tiny $1$} 1, {\tiny $2$} 2, {\tiny $3$} 3, {\tiny $4$} 4, {\tiny $5$} 5}
				
				\axislabels {y}{{\tiny $1$} 1,{\tiny $2$} 2, {\tiny $3$} 3, {\tiny $4$} 4,  {\tiny $-1$} -1, {\tiny $-2$} -2}
				\tlabels{[tc](\xmax, 0){$x$} [cr](0, \ymax){$T(x)$}}
				% Grid
				%\drawcolor[gray]{0.25}
				%\gridlines{1, 1}
				\drawcolor[gray]{0.75} 
				\grid{1,1}
				
				
				
				
			\end{mfpic}
			
		\end{center}
	\end{minipage}
	\begin{minipage}{0.5\linewidth}
		Determine whether the curve is the graph of a function of $x$. Briefly explain your reasoning.
		\end{minipage}

\item Redo Question \ref{vertical} using this graph:

		\begin{center}
			
			
			
			\begin{mfpic}[20]{-1}{6}{-2}{5}
				
				
				\polyline{(0,2), (2,2)} 
				
				\polyline{(2,1), (5,1)}
				
				\point[5pt]{(0,2),   (2,1),(5,3)}
				\circle{(5,1),0.15}
				\circle{(2,2),0.15}
				\axes
				
				\xmarks{1,2,3,4,5}
				
				\ymarks{-2,-1,1,2,3,4,}
				
				\tlpointsep{4pt}
				
				\axislabels {x}{{\tiny $1$} 1, {\tiny $2$} 2, {\tiny $3$} 3, {\tiny $4$} 4, {\tiny $5$} 5}
				
				\axislabels {y}{{\tiny $1$} 1,{\tiny $2$} 2, {\tiny $3$} 3, {\tiny $4$} 4,  {\tiny $-1$} -1, {\tiny $-2$} -2}
				\tlabels{[tc](\xmax, 0){$x$} [cr](0, \ymax){$T(x)$}}
				% Grid
				%\drawcolor[gray]{0.25}
				%\gridlines{1, 1}
				\drawcolor[gray]{0.75} 
				\grid{1,1}
				
				
				
				
			\end{mfpic}
			
		\end{center}
 

\item In example 3 on pg 192, the function $f(x)=4.5t+28$ gives the water level $f(t)$ (in feet) in a reservoir $t$ years after a dam was built. What do the number 4.5 and 28 represent in practical terms?
		

	\item If 
	\begin{equation*}
	y = \begin{cases}
	x-1 & \text{if } x \leq 3,\\
	2 & \text{if } x > 3
	\end{cases}
	\end{equation*}
	is $y$ a function of $x$? Is $x$ a function of $y$? Show evidence of Your thinking.


\item The graph shown below represents a relation (that is, a collection of ordered pairs.) Is $y$ a function of $x$? Is $x$ a function of $y$? 
	\begin{center}
		
		
		
		\begin{mfpic}[20]{-1}{6}{-2}{5}
			
			%\polyline{(0,-2), (4,1), (4,2), (5,3)}
			
			\polyline{(0,2), (2,2)} 
			
			\polyline{(2,1), (5,1)}
			
			\point[5pt]{(0,2),  (2,2)}
			\circle{(2,1),0.15}
			\circle{(5,1),0.15}
			
			
			%\tlabel[cc](-1,1){\scriptsize $(0,1)$}
			
			%\tlabel[cc](2,3.5){\scriptsize $(2,3)$}
			
			%\tlabel[cc](4.5,2.5){\scriptsize $(4,2)$}
			
			%\tlabel[cc](5,-0.5){\scriptsize $(5,0)$}
			
			%\tlabel[cc](6,-0.5){\scriptsize $x$}
			
			%\tlabel[cc](0.5,6){\scriptsize $y$}
			
			
			%\tcaption{\scriptsize $y=f(x)$}
			
			\axes
			
			\xmarks{1,2,3,4,5}
			
			\ymarks{-2,-1,1,2,3,4,}
			
			\tlpointsep{4pt}
			
			\axislabels {x}{{\tiny $1$} 1, {\tiny $2$} 2, {\tiny $3$} 3, {\tiny $4$} 4, {\tiny $5$} 5}
			
			\axislabels {y}{{\tiny $1$} 1,{\tiny $2$} 2, {\tiny $3$} 3, {\tiny $4$} 4,  {\tiny $-1$} -1, {\tiny $-2$} -2}
			
			% Grid
			%\drawcolor[gray]{0.25}
			%\gridlines{1, 1}
			\drawcolor[gray]{0.75} 
			\grid{1,1}
			
		\end{mfpic}
		
	\end{center}

	\item  The function $f(t)$ represents the number of students enrolled in a college course in week $t$ of the Spring 2018 semester. What does $f(2)=30$ mean in practical terms?

\item A home owner mows the lawn every Sunday afternoon. Sketch a rough graph of the height of the grass as a function of time over the course of a two-week period beginning on a Wednesday. Your graph should include units. Briefly explain how you came up with the shape of the graph.

\item Provide an example of a function where the range does not map back to the domain with single values (that is, there is at least one item in the range that is paired with more than one item in the domain).

\item The amount of trash in a county landfill is modeled by the function
$\displaystyle T=f(x) = 180x + 35$ where $x$ is the number of years since 1996 and 
$T(x)$ is measured in thousands of tons.

\begin{enumerate}
	\item What does the number 180 represent in practical terms? What are its units?

	\item  What does the number 35 represent in practical terms?  What are its units?
	
	\item Does the number 7 in $f(7)$ represents: 
	\begin{enumerate}
		\item Thousands of tons, OR 
		\item Number of years since 1996?
	\end{enumerate}
\end{enumerate}

\end{enumerate}	

 \begin{mdframed}[style=exampledefault,frametitle={Looking Ahead to Next Week...}]
 	{\underline{Definitions that you should be familiar with by the next class meeting:} }
 	\begin{enumerate}[label= {  \arabic*:},labelindent=2em, style = standard,leftmargin=4pc, labelsep=*, noitemsep]
 		\item Function;
                     \item Domain range, input, output, linear function.
 		
 		%\item Amplitude, period and midline.
 	\end{enumerate}
 	{\underline{You should be able to:} }
 	\begin{enumerate}[label= {  \arabic*:},labelindent=2em, style = standard,leftmargin=4pc, labelsep=*, noitemsep]
 		\item Be comfortable with the four ways to represent a function (verbal, visual, algebraic and numerical);
 		\item Decide whethera relation represents a function (including the vertical line test, in case the relation is given as a graph);
 		\item Find the domain of a function if a function is given in one of the four ways described above);
                      \item Find function inputs and outputs;
                     \item Find all the inputs for which the function outputs are equal (or less than or equal, or greater than or equal) to a given $y$-value;
                   \item Describe the slope and $y$-intercept of a linear model in practcal terms.
 	\end{enumerate}
{\underline{Advice on how to excel in precalculus:} }
\begin{enumerate}[label= {  \arabic*:},labelindent=2em, style = standard,leftmargin=4pc, labelsep=*, noitemsep]
 		\item  Always go through the discussion questions slowly again yourself after each class, to make sure you fully understand.
Check the  work you did on the questions before class against the answers provided in class to make sure you didn’t overlook anything.
\item Always read the sections we will cover{\bf{ before class}}.
\item Always attempt the discussion questionsr{\bf{ before class}}.
\item Give yourself sufficient time to do the homework: start early and don't just try to guess the answers. Work through each problem and take notes on how you attempt to solve the problem, why your attempt works or where you got stuck, and why.
\end{enumerate}
 	\end{mdframed}


\end{document} 
              