\documentclass[12pt,dvipsnames]{article}
\usepackage[margin=0.4in,footskip=0.1in]{geometry}
\usepackage{etex}
\usepackage{amssymb,amsmath,multicol} %<-- InWorksheetExam1 i also have fancyhdr,
\usepackage{hyperref}
\usepackage[metapost]{mfpic}
\usepackage[pdftex]{graphicx}
\usepackage{csquotes}
\usepackage{pst-plot}
\usepackage{pgfplots}
\pgfplotsset{compat=1.9}

\usepackage{tikz}
\usepackage{tkz-2d}
\usepackage{tkz-base}
\usetikzlibrary{calc}
\usepackage{color}
\usepackage[inline]{enumitem}
\usepackage{refcount}%<-- non in WorksheetExam1

\usepackage[linewidth=1pt]{mdframed}

\usepackage{caption}
\usetikzlibrary{calc,fit,intersections,shapes,calc}
\usetikzlibrary{backgrounds}
\usepackage{systeme}

\newcolumntype{?}{!{\vrule width 1pt}}


%%These three lines are for the typewriter font. Comment them out if I don't want the font.
%%%%%%\renewcommand*\ttdefault{lcmtt}
%%%%%%\renewcommand*\familydefault{\ttdefault} %% Only if the base font of the document is to be typewriter style
%%%%%%\usepackage[T1]{fontenc}
%%%%%%

\usepackage{tabularx, booktabs}


\newenvironment{myitemize}
{ \begin{itemize}
		\setlength{\itemsep}{10pt}
		\setlength{\parskip}{10pt}
		\setlength{\parsep}{10pt}     }
	{ \end{itemize}   
	
} 

\usepackage{setspace}

\font\maxi=cminch scaled 100
\usepackage{tgadventor}
%\renewcommand*\familydefault{\sfdefault} %% Only if the base font of the document is to be sans serif
\usepackage[T1]{fontenc}
\newcommand*{\myfont}{\fontfamily{\sfdefault}\selectfont}
\usepackage{pacioli}
\usepackage[OT1]{fontenc}
\usepackage{systeme}


%\usepackage{spalign}


%\usepackage{AlegreyaSans} %% Option 'black' gives heavier bold face
%% The 'sfdefault' option to make the base font sans serif
%\renewcommand*\oldstylenums[1]{{\AlegreyaSansOsF #1}}

\newcommand*\circled[1]{\tikz[baseline=(char.base)]{%
		\node[shape=circle,fill=blue!20,draw,inner sep=2pt] (char) {#1};}}

\usepackage{lastpage}
\usepackage{fancyhdr}
\pagestyle{fancy} 

\rfoot{{\small{Page \thepage\ of \pageref{LastPage}}}}
\cfoot{}
\renewcommand{\baselinestretch}{1.50}\normalsize



\opengraphsfile{Wk_3_DQ_Fa18}

\begin{document}
\thispagestyle{empty}

%	\thispagestyle{empty}
	\begin{center}
		{\large{Week 3}}
	\end{center}

{\bfseries{Textbook sections to read and annotate before class:}}  2.6.
\smallskip

	{\bfseries{Definitions to memorize before class:}} function;  domain; range, $x$-intercept; $y$-intercept.
\smallskip	
	
{\bfseries{Skills to review before class:} }
	\begin{enumerate} 

		\item How to find the domain and range of a function from a graph;
		\item How to find the domain of a function given by a formula;
\item How to find the domain of a function given by a table;
		\item How to write the domain and range in interval forms, using parentheses, brackets and curly braces;
		\item How to write the formula for a piecewise defined function starting from a graph;
		\item How to translate an application problem into a piecewise defined function;
		\item (In application problems:) How to interpret the slope and intercepts of a linear function in practical terms.
		
\end{enumerate}
	
		
{\bfseries{Bring to class:} } A paper notebook with your annotations of the reading and your work on the questions listed below; a pen and/or a sharpened pencil and an eraser.

{\bfseries{Laptops/Phones Policy:}}  No devices in class, unless the assignment requires it.

{\bfseries{Audio-Recording:}} I will be calling people (by name) from the class roster to go over the discussion questions: to ensure everyone's privacy, please do not audio-record the class.


\begin{center}

{\large{\bfseries{Reading and Discussion Questions for Week 3} }}
\end{center}

\begin{enumerate}[label=\protect\circled{\arabic*}]
		\item Describe (in words, without using formulas) the function transformations described in the reading for today.
\item Starting with a function $f(x)$ and a number $c\not= 0$, describe (in words, without using formulas) how the graph of $f(x)+c$ relates to the graph of $f$. Using the explanations on pg 198 as an example, make sure that your description is as accurate as possible. Some of the questions that your explanation needs to address: 
\begin{enumerate}
	\item How do the domain and range of $f(x)+c$ relate to the domain and range of $f(x)$?
	\item How does the shape of the graph of $f(x)+c$ differ (or is similar to) the graph of $f(x)$?
	\item Does the shape depend on whether $c$ is positive or negative? Why? Why not?
\end{enumerate}
\item \label{quest:add} Starting with a function $f(x)$ and a number $c\not= 0$, describe (in words, without using formulas) how the graph of $f(x+c)$ relates to the graph of $f$. Using the explanations on pg 199 as an example, make sure that your description is as accurate as possible. Some of the questions that your explanation needs to address: 
\begin{enumerate}
	\item How do the domain and range of $f(x+c)$ relate to the domain and range of $f(x)$?
	\item How does the shape of the graph of $f(x+c)$ differ (or is similar to) the graph of $f(x)$?
	\item Does the shape depend on whether $c$ is positive or negative? Why? Why not?
\end{enumerate}

\item The graph of two functions $f(x)$ and $g(x)$ are shown below. What is the formula for $g(x)$?


\begin{minipage}[t]{0.5\linewidth}
	\begin{tikzpicture}[scale=0.65]
	\begin{axis}[
	domain=-2:4,
	axis y line=center,
	axis x line=middle,
	]
	%\addplot [no markers,line width=1pt,<->] {x^2} node [pos=0.3,left] {$f(x)=x^2$};
	\addplot [no markers,line width=1pt,<->]{x^2}
	%node[pos=0.1,pin=135:{\color{purple}$f(x)=-2$}] {}
	node [pos=0.8,pin={200:$f(x)=x^2$},inner sep=0pt] {};
	%node[pos=0.9,pin=135:{\color{green!70!black}$f(x)=3$}] {}
	;
	\addplot [no markers,line width=1pt,<->]{(x-2)^2}
	%node[pos=0.1,pin=135:{\color{purple}$f(x)=-2$}] {}
	node[pos=0.99,pin=200:{\color{blue}$g(x)$},inner sep=0pt] {}
	%node[pos=0.9,pin=135:{\color{green!70!black}$f(x)=3$}] {}
	;
	\end{axis}
	\end{tikzpicture}
\end{minipage}%
\begin{minipage}[b]{0.5\linewidth}
	\begin{enumerate}
		\item $g(x)=x^2+2$
		\item $g(x)=x^2-2$
		\item $g(x)=(x+2)^2$
		\item $g(x)=(x-2)^2$
	\end{enumerate}
\end{minipage}


		\item \label{item:key} The graph of a function $f(x)$ is shown below.
		
		\begin{center}
			
			\begin{mfpic}[20]{-1}{6}{-2}{5}
				\polyline{(0,2), (1,4)} 
				\polyline{(1,4), (2,4)}
				\polyline{(2,4), (4,0)}
				
				\point[5pt]{(0,2), (4,0)}
				%\circle{(4, 2),0.15}
				\tcaption{\scriptsize $y=f(x)$}
				
				\axes
				\xmarks{1,2,3,4,5}
				\ymarks{-2,-1,1,2,3,4,}
				
				\tlpointsep{4pt}
				
				\axislabels {x}{{\tiny $1$} 1, {\tiny $2$} 2, {\tiny $3$} 3, {\tiny $4$} 4, {\tiny $5$} 5}
				
				\axislabels {y}{{\tiny $1$} 1,{\tiny $2$} 2, {\tiny $3$} 3, {\tiny $4$} 4,  {\tiny $-1$} -1, {\tiny $-2$} -2}
				
				\drawcolor[gray]{0.75} 
				\grid{1,1}
				
			\end{mfpic}
			
		\end{center}
		
		\begin{enumerate}[label=\textcolor{blue}{\bf (\alph*)}]
			
			\item Fill out the table:
			
			
			\begin{minipage}{\linewidth}
				\centering
				\captionof{table}{} \label{tab:title} 
				\begin{tabular}{|l|l|l|l|l|l|l|l|}
					\hline
					$x$    & $-1$ & $0$ & $1$ & $2$ & $3$ & $4$ & $5$ \\ \hline
					$f(x)$ &      &     &     &     &     &     &     \\ \hline
				\end{tabular}
			\end{minipage}
			
			
			
			\item What are the domain and range of $f$? %Write the formula for $f$ (hint: it is a piecewise defined function with three parts.)
			
			
			\item \label{ettob12} Fill out the table:
			
			\begin{minipage}{\linewidth}
				\centering
				\captionof{table}{} \label{tab:ettobbotte}  
				\begin{tabular}{|l|l|l|l|l|l|l|l|l|l|}
					\hline
					$x$    & $-1$ & $0$ & 0.5 & $1$ & 1.5 & $2$ & $3$ & $4$ & $5$ \\ \hline
					$f(x+1)$ &      &     &     &     &     &     &   &  & \\ \hline
				\end{tabular}
			\end{minipage}
			
			Note that if you start  with an $x$-value, for example $x=2$, the table asks you to pair this $x$ value with $f(2+1)$, which is 2 (refer to the graph of $f(x)$: in this graph, when $x=3$, then $f(3)=2$.)
			\item \label{parapero12} Plot the points from table \ref{tab:ettobbotte}. Make sure that the overall shape of your new graph is consistent with your answers in Question \ref{quest:add}
			\item What are the domain and range of your new graph?
		\end{enumerate}
		

			\item Starting with a function $f(x)$, describe (in words, without using formulas) how the graphs of $f(-x)$ and $-f(x)$ relate to the graph of $f$. Using the explanations on pg 201 as an example, make sure that your description is as accurate as possible. Some of the questions that your explanation needs to address: 
			\begin{enumerate}
				\item How do the domain and range of $f(-x)$ relate to the domain and range of $f(x)$?
					\item How do the domain and range of $-f(x)$ relate to the domain and range of $f(x)$?
		\end{enumerate}
		
		\item If the function $f(x)$ is given by the formula $\displaystyle f(x)=x^2+4$, write a formula for $f(-x)$ and a formula for $-f(x)$.
		
		\item  The transformations
		
		\begin{enumerate}
			\item Shift left 2 units;
			\item Reflection across the $y$-axis;
			\item Shift down 3 unit
		\end{enumerate}
		are applied to the graph of $\displaystyle y=\lvert x \rvert$ ({\underline {in the given order}}).
		\begin{enumerate} 
			\item  Write the equation for the final transformed graph $g(x)$.
			\item The range of $\displaystyle y=\lvert x \rvert$ is $[0,\infty)$. Find the range of $g$ and explain your reasoning.
			\item Would the graph of $g$ remain the same if the changed the order of the transformations? Why? Why not?
		\end{enumerate}
			
\item \label{quest:multiply} Starting with a function $f(x)$ and a number $c> 0, c\not = 1$, describe (in words, without using formulas) how the graph of $f(cx)$ relates to the graph of $f$. Using the explanations on pg 203 as an example, make sure that your description is as accurate as possible. Some of the questions that your explanation needs to address: 
\begin{enumerate}
	\item How do the domain and range of $f(cx)$ relate to the domain and range of $f(x)$?
	\item How does the shape of the graph of $f(cx)$ differ (or is similar to) the graph of $f(x)$?
	\item Does the shape depend on whether $0<c<1$ or $c>1$? Why? Why not? Please explain Your reasoning verbally and with graphs. 
\end{enumerate}			
			
			\item Redo parts \ref{ettob12} and \ref{parapero12} of Question \ref{item:key} using the table:
			
			\begin{minipage}{\linewidth}
				\centering
				\captionof{table}{} \label{tab:ettob2}  
				\begin{tabular}{|l|l|l|l|l|l|l|l|l|l|}
					\hline
					$x$    & $0$ & $0.25$ & 0.5 & $0.75$ & 1 & $1.25$ & $1.5$ & $1.75$ & $2$ \\ \hline
					$f(2x)$ &      &     &     &     &     &     &   &  & \\ \hline
				\end{tabular}
			\end{minipage}
			
	The graph of the function $f(x)$ is included below for Your convenience.
	
	\begin{center}
		
		\begin{mfpic}[20]{-1}{6}{-2}{5}
			\polyline{(0,2), (1,4)} 
			\polyline{(1,4), (2,4)}
			\polyline{(2,4), (4,0)}
			
			\point[5pt]{(0,2), (4,0)}
			%\circle{(4, 2),0.15}
			\tcaption{\scriptsize $y=f(x)$}
			
			\axes
			\xmarks{1,2,3,4,5}
			\ymarks{-2,-1,1,2,3,4,}
			
			\tlpointsep{4pt}
			
			\axislabels {x}{{\tiny $1$} 1, {\tiny $2$} 2, {\tiny $3$} 3, {\tiny $4$} 4, {\tiny $5$} 5}
			
			\axislabels {y}{{\tiny $1$} 1,{\tiny $2$} 2, {\tiny $3$} 3, {\tiny $4$} 4,  {\tiny $-1$} -1, {\tiny $-2$} -2}
			
			\drawcolor[gray]{0.75} 
			\grid{1,1}
			
		\end{mfpic}
		
	\end{center}
\item A table of values for a function $h(x)$ is shown below:

\begin{tabularx}{0.8\textwidth}{?X?X?X?X?X?X?X?X?X?X?}
	\hline
	$x$ & $1$ & $2$ & $3$ & $4$ & $5$ & $6$ & $7$ & $8$ & $9$\\ \hline
	$h(x)$ & $0.5$ & $1$ & $0.5$ & $2$ & $3$ & $1$ & $3$ & $4$ & $3$  \\ \hline
\end{tabularx}

Fill the table:

\begin{tabularx}{0.8\textwidth}{?X?X?X?X?X?}
	\hline
	$x$ & $1$ & $2$ & $3$ & $4$  \\ \hline
	$h(2x+1)$ &  &  &  &   \\ \hline
\end{tabularx}	

\item Suppose that $y=t(c)$ represents the function in which $c$ is the number of credits that a student is taking this year and  $y=t(c)$ is the cost of tuition (in dollars). If the cost per credit has increased by 2\% from last year, what function represents the new cost of tuition?

\begin{enumerate}
	\item $y=t(1.02c)$;
	\item $y=t(c+1.02)$;
	\item $y=t(c)+1.02$;
	\item $y=1.02t(c)$.
\end{enumerate}		

\item Suppose that $y=t(c)$ represents the function in which $c$ is the number of credits that a student is taking this year $y=t(c)$ is the cost of tuition (in dollars). Suppose a change was made to the tuition so that $y=100+t(c)$ describes the new tuition function. Which of the following situations best describes the new tuition function?

\begin{enumerate}
	\item The number of credit hours has increased by 100;
	\item The cost per credit has increased by \$100;
	\item The tuition has increased by 100\%;
	\item The total cost of tuition has increased by \$100.
\end{enumerate}
	
		%	\item Redo Questions \ref{ettob12} and \ref{parapero12} for $\displaystyle f\left (\frac{x}{2}\right )$, $f(-x)$ and $-f(x)$. Make sure to choose a sufficient number of $x$ values in your tables so that when you plot the points you can visualize the main features of the graphs.
	%	\end{enumerate}
		
	%			\item If we apply two transformations to a function, does the order in which we perform these transformations matter? If you think that the order does matter, draw the graph of a function and give two transformations that will result in two different graphs depending on the order in which you apply them. If you think that the order does not matter, explain why. [Note: you are not required to write a formula for the initial function.]
		%\newpage
		
		\item The graph of a function $g(x)$ is shown below.
		
		
		\pgfmathdeclarefunction{f}{1}{%
			\pgfmathparse{#1*(-0.25)+1}%
		}
		
		\fbox{\begin{tikzpicture}[baseline=(current bounding box.north),scale=0.5]
			\begin{axis}[
			axis y line=center,
			axis x line=middle, 
			xmin=0,
			xmax=4,
			ymin=0,
			ymax=4,
			xlabel=\scalebox{1.5}{$x$},
			ylabel=$y$,
			x label style={at={(current axis.right of origin)},anchor=north, below=5mm},
			y label style={at={(current axis.above origin)},rotate=0,anchor=south east,left=5mm},
			clip=false,
			grid=both,
			minor xtick={0,1,...,4},
			minor ytick={1,2,...,4},
			enlarge x limits=0,
			scaled x ticks = true
			]
			\addplot[domain=0:4,blue,line width=2.0pt] {f(x)}; %no shift (clearly centered
			%\addplot {f(x-1)}; %little shift to the right
			%\addplot {f(2*x)}; %shifted nearly off the sheet
			\draw [draw=blue, fill=blue, thick] (axis cs: 0, 1) circle (5.0pt);
			\draw [draw=blue, fill=blue, thick] (axis cs: 4, 0) circle (5.0pt);
			\end{axis}
			\node[below, yshift=-4mm] at (current bounding box.south) {Function $g(x)$};
			\end{tikzpicture}
		}     
		\parbox[t]{11cm}{\vskip0pt
			Explain how the graph of $g$ must be changed in order to get each of the graphs shown below. (Sample answer: The graph in Fig. A is obtained by moving the graph of $g$ by 1 unit to the right.)
			
			
			
		}
		
		%%%%%%%%%%%%%%
		
		\pgfplotsset{
			standard/.style={
				axis x line=middle,
				axis y line=middle,
				enlarge x limits=0.15,
				enlarge y limits=0.15,
				every axis x label/.style={at={(current axis.right of origin)},anchor=north west},
				every axis y label/.style={at={(current axis.above origin)},anchor=north east}
			}
		}
		
		\begin{figure}[h]
			\hspace*{\fill}
			\begin{tikzpicture}[scale=0.5]
			\begin{axis}[
			axis y line=center,
			axis x line=middle, 
			xmin=0,
			xmax=5,
			ymin=0,
			ymax=4,
			xlabel=\scalebox{1.5}{$x$},
			ylabel=\scalebox{1.5}{$y$},
			%x label style={at={(current axis.right of origin)},anchor=north, below=5mm},
			%y label style={at={(current axis.above origin)},rotate=0,anchor=south east,left=5mm},
			clip=false,
			grid=both,
			minor xtick={0,1,...,4},
			minor ytick={1,2,...,4},
			enlarge x limits=0,
			scaled x ticks = true
			]
			\addplot[domain=1:5,blue,line width=2.0pt] {f(x-1)}; %no shift (clearly centered
			%\addplot {f(x-1)}; %little shift to the right
			%\addplot {f(2*x)}; %shifted nearly off the sheet
			\draw [draw=blue, fill=blue, thick] (axis cs: 1, 1) circle (5.0pt);
			\draw [draw=blue, fill=blue, thick] (axis cs: 5, 0) circle (5.0pt);
			\end{axis}
			%\end{tikzpicture} 
			\node[below, yshift=-4mm] at (current bounding box.south) {Fig. A};
			\end{tikzpicture}\hspace*{\fill}
			\begin{tikzpicture}[scale=0.5]
			\begin{axis}[
			axis y line=center,
			axis x line=middle, 
			xmin=-1,
			xmax=4,
			ymin=0,
			ymax=4,
			xlabel=\scalebox{1.5}{$x$},
			ylabel=\scalebox{1.5}{$y$},
			%axis labels at tip,
			%x label style={at={(current axis.right of origin)},anchor=north, below=5mm},
			%y label style={at={(current axis.above origin)},rotate=0,anchor=north,left=5mm,yshift=1.5ex},
			clip=false,
			grid=both,
			minor xtick={0,1,...,4},
			minor ytick={1,2,...,4},
			enlarge x limits=0,
			scaled x ticks = true
			]
			\addplot[domain=-1:3,blue,line width=2.0pt] {f(x+1)}; %no shift (clearly centered
			%\addplot {f(x-1)}; %little shift to the right
			%\addplot {f(2*x)}; %shifted nearly off the sheet
			\draw [draw=blue, fill=blue, thick] (axis cs: -1, 1) circle (5.0pt);
			\draw [draw=blue, fill=blue, thick] (axis cs: 3, 0) circle (5.0pt);
			\end{axis}
			%\end{tikzpicture} 
			\node[below, yshift=-4mm] at (current bounding box.south) {Fig. B};
			\end{tikzpicture}\hspace*{\fill}
			\begin{tikzpicture}[scale=0.5]
			\begin{axis}[
			axis y line=center,
			axis x line=middle, 
			xmin=-4,
			xmax=0,
			ymin=0,
			ymax=4,
			xlabel=\scalebox{1.5}{$x$},
			ylabel=\scalebox{1.5}{$y$},
			%axis labels at tip,
			%x label style={at={(current axis.right of origin)},anchor=north, below=5mm},
			%y label style={at={(current axis.above origin)},rotate=0,anchor=north,left=5mm,yshift=1.5ex},
			clip=false,
			grid=both,
			minor xtick={-4,-3,...,-1},
			minor ytick={1,2,...,4},
			enlarge x limits=0,
			scaled x ticks = true
			]
			\addplot[domain=-4:0,blue,line width=2.0pt] {f(-x)}; %no shift (clearly centered
			%\addplot {f(x-1)}; %little shift to the right
			%\addplot {f(2*x)}; %shifted nearly off the sheet
			\draw [draw=blue, fill=blue, thick] (axis cs: -4, 0) circle (5.0pt);
			\draw [draw=blue, fill=blue, thick] (axis cs: 0, 1) circle (5.0pt);
			\end{axis}
			%\end{tikzpicture} 
			\node[below, yshift=-4mm] at (current bounding box.south) {Fig. C};
			\end{tikzpicture}\hspace*{\fill}
			%\caption{Two figures side by-side}
			%\label{fig:test}
		\end{figure}
		%%%%%%%%%%%%%%%%%%%%
		\begin{figure}[h]
			\hspace*{\fill}
			\begin{tikzpicture}[scale=0.5]
			\begin{axis}[
			axis y line=center,
			axis x line=middle, 
			xmin=0,
			xmax=4,
			ymin=0,
			ymax=2,
			xlabel=\scalebox{1.5}{$x$},
			ylabel=\scalebox{1.5}{$y$},
			%x label style={at={(current axis.right of origin)},anchor=north, below=5mm},
			%y label style={at={(current axis.above origin)},rotate=0,anchor=south east,left=5mm},
			clip=false,
			grid=both,
			minor xtick={1,2,3,4},
			minor ytick={1,2},
			enlarge x limits=0,
			scaled x ticks = true
			]
			\addplot[domain=0:4,blue,line width=2.0pt] {0.5*f(x)}; %no shift (clearly centered
			%\addplot {f(x-1)}; %little shift to the right
			%\addplot {f(2*x)}; %shifted nearly off the sheet
			\draw [draw=blue, fill=blue, thick] (axis cs: 0, 0.5) circle (5.0pt);
			\draw [draw=blue, fill=blue, thick] (axis cs: 4, 0) circle (5.0pt);
			\end{axis}
			%\end{tikzpicture} 
			\node[below, yshift=-4mm] at (current bounding box.south) {Fig. D};
			\end{tikzpicture}\hspace*{\fill}
			\begin{tikzpicture}[scale=0.5]
			\begin{axis}[
			axis y line=center,
			axis x line=middle, 
			xmin=0,
			xmax=4,
			ymin=0,
			ymax=6,
			xlabel=\scalebox{1.5}{$x$},
			ylabel=\scalebox{1.5}{$y$},
			%axis labels at tip,
			%x label style={at={(current axis.right of origin)},anchor=north, below=5mm},
			%y label style={at={(current axis.above origin)},rotate=0,anchor=north,left=5mm,yshift=1.5ex},
			clip=false,
			grid=both,
			minor xtick={0,1,...,4},
			minor ytick={1,2,...,6},
			enlarge x limits=0,
			scaled x ticks = true
			]
			\addplot[domain=0:4,blue,line width=2.0pt] {2*f(x)}; %no shift (clearly centered
			%\addplot {f(x-1)}; %little shift to the right
			%\addplot {f(2*x)}; %shifted nearly off the sheet
			\draw [draw=blue, fill=blue, thick] (axis cs: 0, 2) circle (5.0pt);
			\draw [draw=blue, fill=blue, thick] (axis cs: 4, 0) circle (5.0pt);
			\end{axis}
			%\end{tikzpicture} 
			\node[below, yshift=-4mm] at (current bounding box.south) {Fig. E};
			\end{tikzpicture}\hspace*{\fill}
			\begin{tikzpicture}[scale=0.5]
			\begin{axis}[
			axis y line=center,
			axis x line=middle, 
			xmin=0,
			xmax=2,
			ymin=0,
			ymax=4,
			xlabel=\scalebox{1.5}{$x$},
			ylabel=\scalebox{1.5}{$y$},
			%axis labels at tip,
			%x label style={at={(current axis.right of origin)},anchor=north, below=5mm},
			%y label style={at={(current axis.above origin)},rotate=0,anchor=north,left=5mm,yshift=1.5ex},
			clip=false,
			grid=both,
			minor xtick={0,1,2},
			minor ytick={1,2,...,4},
			enlarge x limits=0,
			scaled x ticks = true
			]
			\addplot[domain=0:2,blue,line width=2.0pt] {f(2*x)}; %no shift (clearly centered
			%\addplot {f(x-1)}; %little shift to the right
			%\addplot {f(2*x)}; %shifted nearly off the sheet
			\draw [draw=blue, fill=blue, thick] (axis cs: 0, 1) circle (5.0pt);
			\draw [draw=blue, fill=blue, thick] (axis cs: 2, 0) circle (5.0pt);
			\end{axis}
			%\end{tikzpicture} 
			\node[below, yshift=-4mm] at (current bounding box.south) {Fig. F};
			
			\end{tikzpicture}\hspace*{\fill}
			%\caption{Two figures side by-side}
			%\label{fig:test}
		\end{figure}
		%%%%%%%%%%%%%%%%%%%%
		\begin{figure}[h]
			\hspace*{\fill}
			\begin{tikzpicture}[scale=0.5]
			\begin{axis}[
			axis y line=center,
			axis x line=middle, 
			xmin=0,
			xmax=4,
			ymin=-3,
			ymax=1,
			xlabel=\scalebox{1.5}{$x$},
			ylabel=\scalebox{1.5}{$y$},
			%x label style={at={(current axis.right of origin)},anchor=north, below=5mm},
			%y label style={at={(current axis.above origin)},rotate=0,anchor=south east,left=5mm},
			clip=false,
			grid=both,
			minor xtick={1,2,3,4},
			minor ytick={-3,-2,-1,1},
			enlarge x limits=0,
			scaled x ticks = true
			]
			\addplot[domain=0:4,blue,line width=2.0pt] {-f(x)}; %no shift (clearly centered
			%\addplot {f(x-1)}; %little shift to the right
			%\addplot {f(2*x)}; %shifted nearly off the sheet
			\draw [draw=blue, fill=blue, thick] (axis cs: 0, -1) circle (5.0pt);
			\draw [draw=blue, fill=blue, thick] (axis cs: 4, 0) circle (5.0pt);
			\end{axis}
			%\end{tikzpicture} 
			\node[below, yshift=-4mm] at (current bounding box.south) {Fig. G};
			\end{tikzpicture}\hspace*{\fill}
			\begin{tikzpicture}[scale=0.5]
			\begin{axis}[
			axis y line=center,
			axis x line=middle, 
			xmin=0,
			xmax=8,
			ymin=0,
			ymax=3,
			xlabel=\scalebox{1.5}{$x$},
			ylabel=\scalebox{1.5}{$y$},
			%axis labels at tip,
			%x label style={at={(current axis.right of origin)},anchor=north, below=5mm},
			%y label style={at={(current axis.above origin)},rotate=0,anchor=north,left=5mm,yshift=1.5ex},
			clip=false,
			grid=both,
			minor xtick={0,2,...,8},
			minor ytick={1,2,3},
			enlarge x limits=0,
			scaled x ticks = true
			]
			\addplot[domain=0:8,blue,line width=2.0pt] {f(0.5*x)}; %no shift (clearly centered
			%\addplot {f(x-1)}; %little shift to the right
			%\addplot {f(2*x)}; %shifted nearly off the sheet
			\draw [draw=blue, fill=blue, thick] (axis cs: 0, 1) circle (5.0pt);
			\draw [draw=blue, fill=blue, thick] (axis cs: 8, 0) circle (5.0pt);
			\end{axis}
			%\end{tikzpicture} 
			\node[below, yshift=-4mm] at (current bounding box.south) {Fig. H};
			\end{tikzpicture}\hspace*{\fill}
			\begin{tikzpicture}[scale=0.5]
			\begin{axis}[
			axis y line=center,
			axis x line=middle, 
			xmin=0,
			xmax=4,
			ymin=0,
			ymax=4,
			xlabel=\scalebox{1.5}{$x$},
			ylabel=\scalebox{1.5}{$y$},
			%axis labels at tip,
			%x label style={at={(current axis.right of origin)},anchor=north, below=5mm},
			%y label style={at={(current axis.above origin)},rotate=0,anchor=north,left=5mm,yshift=1.5ex},
			clip=false,
			grid=both,
			minor xtick={0,1,2,3,4},
			minor ytick={1,2,...,4},
			enlarge x limits=0,
			scaled x ticks = true
			]
			\addplot[domain=0:4,blue,line width=2.0pt] {f(x)+1}; %no shift (clearly centered
			%\addplot {f(x-1)}; %little shift to the right
			%\addplot {f(2*x)}; %shifted nearly off the sheet
			\draw [draw=blue, fill=blue, thick] (axis cs: 0, 2) circle (5.0pt);
			\draw [draw=blue, fill=blue, thick] (axis cs: 4, 1) circle (5.0pt);
			\end{axis}
			%\end{tikzpicture} 
			\node[below, yshift=-4mm] at (current bounding box.south) {Fig. L};
			\end{tikzpicture}\hspace*{\fill}
			%\caption{Two figures side by-side}
			%\label{fig:test}
		\end{figure}
		
		The equation of $g(x)$ is: $\displaystyle g(x)=1-\frac{x}{4}$. Can you write the equations of (some of) the graphs in Fig. A through L? 
		
		

\end{enumerate}	

 \begin{mdframed}[style=exampledefault,frametitle={Looking Ahead to Next Week...}]
 	{\underline{Definitions that you should be familiar with by the next class meeting:} }
 	\begin{enumerate}[label= {  \arabic*:},labelindent=2em, style = standard,leftmargin=4pc, labelsep=*, noitemsep]
 		\item Horizontal stretch and compression of a function;
                     \item Horizontal and vertical shift of a function;
                     \item Reflection around the $x$-axis or the $y$-axis.
 		
 		%\item Amplitude, period and midline.
 	\end{enumerate}
 	{\underline{You should be able to:} }
 	\begin{enumerate}[label= {  \arabic*:},labelindent=2em, style = standard,leftmargin=4pc, labelsep=*, noitemsep]
 		\item Identify function transformations from the formula for a function (for example, if $f(x)$ is given, what transformation is represented by $f(2x)$?);
                     \item Write a table of values for function transformations of a function;
                     \item Identify the domain and range of function transformations of a function.
 	\end{enumerate}
{\underline{Advice on how to excel in precalculus:} }
\begin{enumerate}[label= {  \arabic*:},labelindent=2em, style = standard,leftmargin=4pc, labelsep=*, noitemsep]
 		\item  Always go through the discussion questions slowly again yourself after each class, to make sure you fully understand.
Check the  work you did on the questions before class against the answers provided in class to make sure you didn’t overlook anything.
\item Always read the sections we will cover{\bf{ before class}}.
\item Always attempt the discussion questionsr{\bf{ before class}}.
\item Give yourself sufficient time to do the homework: start early and don't just try to guess the answers. Work through each problem and take notes on how you attempt to solve the problem, why your attempt works or where you got stuck, and why.
\end{enumerate}
 	\end{mdframed}


\end{document} 
              