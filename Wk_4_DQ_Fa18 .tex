\documentclass[12pt,dvipsnames]{article}
\usepackage[margin=0.4in,footskip=0.1in]{geometry}
\usepackage{etex}
\usepackage{amssymb,amsmath,multicol} %<-- InWorksheetExam1 i also have fancyhdr,
\usepackage{hyperref}
\usepackage[metapost]{mfpic}
\usepackage[pdftex]{graphicx}
\usepackage{csquotes}
\usepackage{pst-plot}
\usepackage{pgfplots}
\pgfplotsset{compat=1.9}

\usepackage{tikz}
\usepackage{tkz-2d}
\usepackage{tkz-base}
\usetikzlibrary{calc}
\usepackage{color}
\usepackage[inline]{enumitem}
\usepackage{refcount}%<-- non in WorksheetExam1

\usepackage[linewidth=1pt]{mdframed}

\usepackage{caption}
\usetikzlibrary{calc,fit,intersections,shapes,calc}
\usetikzlibrary{backgrounds}
\usepackage{systeme}

\newcolumntype{?}{!{\vrule width 1pt}}


%%These three lines are for the typewriter font. Comment them out if I don't want the font.
%%%%%%\renewcommand*\ttdefault{lcmtt}
%%%%%%\renewcommand*\familydefault{\ttdefault} %% Only if the base font of the document is to be typewriter style
%%%%%%\usepackage[T1]{fontenc}
%%%%%%

\usepackage{tabularx, booktabs}


\newenvironment{myitemize}
{ \begin{itemize}
		\setlength{\itemsep}{10pt}
		\setlength{\parskip}{10pt}
		\setlength{\parsep}{10pt}     }
	{ \end{itemize}   
	
} 

\usepackage{setspace}

\font\maxi=cminch scaled 100
\usepackage{tgadventor}
%\renewcommand*\familydefault{\sfdefault} %% Only if the base font of the document is to be sans serif
\usepackage[T1]{fontenc}
\newcommand*{\myfont}{\fontfamily{\sfdefault}\selectfont}
\usepackage{pacioli}
\usepackage[OT1]{fontenc}
\usepackage{systeme}


%\usepackage{spalign}


%\usepackage{AlegreyaSans} %% Option 'black' gives heavier bold face
%% The 'sfdefault' option to make the base font sans serif
%\renewcommand*\oldstylenums[1]{{\AlegreyaSansOsF #1}}

\newcommand*\circled[1]{\tikz[baseline=(char.base)]{%
		\node[shape=circle,fill=blue!20,draw,inner sep=2pt] (char) {#1};}}

\usepackage{lastpage}
\usepackage{fancyhdr}
\pagestyle{fancy} 

\rfoot{{\small{Page \thepage\ of \pageref{LastPage}}}}
\cfoot{}
\renewcommand{\baselinestretch}{1.50}\normalsize



\opengraphsfile{Wk_4_DQ_Fa18}

\begin{document}
\thispagestyle{empty}

%	\thispagestyle{empty}
	\begin{center}
		{\large{Week 4}}
	\end{center}

{\bfseries{Textbook sections to read and annotate before class:}}  2.7 and 2.8.
\smallskip

	{\bfseries{Definitions to memorize before class:}} function;  domain; range, $x$-intercept; $y$-intercept,  interval form, transformation.
\smallskip	
	
{\bfseries{Skills to review before class:} }
	\begin{enumerate} 
		\item Describe a function transformation using words, with a graph and with a formula;
\item Identify the domain and range of a function transformation;
\item Fill out a table of values for a transformation of a function $f$, starting from the graph or the formula for $f$.
		
\end{enumerate}
	
		
{\bfseries{Bring to class:} } A paper notebook with your annotations of the reading and your work on the questions listed below; a pen and/or a sharpened pencil and an eraser.

{\bfseries{Laptops/Phones Policy:}}  No devices in class, unless the assignment requires it.

{\bfseries{Audio-Recording:}} I will be calling people (by name) from the class roster to go over the discussion questions: to ensure everyone's privacy, please do not audio-record the class.


\begin{center}

{\large{\bfseries{Reading and Discussion Questions for Week 4} }}
\end{center}

\begin{enumerate}[label=\arabic*., leftmargin=2\parindent,
labelindent=\parindent, labelsep=*]	

\item \label{item:one1} On pg. 210, your book starts with two functions, $\displaystyle f(x)=\frac{1}{x-2}$ and $\displaystyle g(x)=\sqrt{x}$, and introduces the four operations (addition, subtraction, multiplication and division). If $A$ is the domain of $f$ and $B$ is the domain of $g$, what are the domains of $\displaystyle f+g, f-g,fg,\frac{g}{f}, \frac{f}{g}$?

\item  Redo Problem \ref{item:one1} with $\displaystyle f(x)=\frac{1}{x-2}$ and $\displaystyle g(x)=\frac{1}{\sqrt{x}}$.
\item The graph of two functions, $f(x)$ and $g(x)$, are shown below.

%\end{mdframed}



\begin{minipage}{0.5\linewidth}
\begin{center}

\begin{mfpic}[20]{-3}{6}{-2}{5}

\polyline{(1,0), (5,2)} 
%\polyline{(2,0), (4,2)}
\point[5pt]{(1,0), (5,2)}
\tcaption{\scriptsize $y=f(x)$}
\axes
\xmarks{-2,-1,1,2,3,4,5}
\ymarks{-2,-1,1,2,3,4,}
\tlpointsep{4pt}
\axislabels {x}{{\tiny $-2$} -2,{\tiny $-1$} -1,{\tiny $1$} 1, {\tiny $2$} 2, {\tiny $3$} 3, {\tiny $4$} 4, {\tiny $5$} 5}
\axislabels {y}{{\tiny $1$} 1,{\tiny $2$} 2, {\tiny $3$} 3, {\tiny $4$} 4,  {\tiny $-1$} -1, {\tiny $-2$} -2}
\drawcolor[gray]{0.75} 
\grid{1,1}
\end{mfpic}
\end{center}
\end{minipage}
\begin{minipage}{0.5\linewidth}
\begin{center}

\begin{mfpic}[20]{-1}{6}{-2}{5}

%\polyline{(0,-2), (4,1), (4,2), (5,3)}

\polyline{(0,2), (4,0)} 

%\polyline{(3,3), (5,4)}

\point[5pt]{(0,2), (4,0)}
%\circle{(3, 3),0.15}
\tcaption{\scriptsize $y=g(x)$}
\axes

\xmarks{1,2,3,4,5}

\ymarks{-2,-1,1,2,3,4,}

\tlpointsep{4pt}

\axislabels {x}{{\tiny $1$} 1, {\tiny $2$} 2, {\tiny $3$} 3, {\tiny $4$} 4, {\tiny $5$} 5}

\axislabels {y}{{\tiny $1$} 1,{\tiny $2$} 2, {\tiny $3$} 3, {\tiny $4$} 4,  {\tiny $-1$} -1, {\tiny $-2$} -2}

\drawcolor[gray]{0.75} 
\grid{1,1}

\end{mfpic}
\end{center}
\end{minipage}




	

	

\begin{enumerate}[labelindent=\parindent,leftmargin=*]
%\begin{enumerate}[label=\protect\circled{\arabic*}]
	\item  Find the domain and range of $f$ and $g$, and write them in interval form.
	

	\item Write the formulas for $f$ and $g$.
	
	\item \label{item:one} Fill out the table:


\begin{minipage}{\linewidth}
\centering
\captionof{table}{$f(x)+g(x)$} 
\begin{tabularx}{0.8\textwidth}{|X|X|X|X|X|X|X|X|}
\hline
\multicolumn{2}{|c|}{$x$}         & $0$ & $1$ & $2$ & $3$ & $4$ & 5  \\ \hline
\multicolumn{2}{|c|}{$f(x)+g(x)$}   & & &     &     &    &         \\ \hline
\end{tabularx}
\end{minipage}

\item \label{item:two} What is the domain of $f+g$?
	
	\item \label{item:three} Write the formula for $f+g$ and graph this function.
	
	\item Redo parts \ref{item:one}, \ref{item:two} and \ref{item:three} replacing $f+g$ with  $\displaystyle \frac{f}{g}$. How does the domain of $\displaystyle \frac{f}{g}$ compare with the domain of $f+g$? 
	
\item Redo parts \ref{item:one}, \ref{item:two} and \ref{item:three} replacing $f+g$ with  $\displaystyle \frac{g}{f}$. How does the domain of $\displaystyle \frac{g}{f}$ compare with the domain of $f+g$? How does the domain of $\displaystyle \frac{g}{f}$ compare with the domain of $\displaystyle \frac{f}{g}$?


		
%\end{enumerate}
		

		
		
		
	%\begin{enumerate}[label=\protect\circled{\arabic*},resume]	
	

	

\end{enumerate}

\item On pg.~213, your textbook states that, if $f$ and $g$ are two functions, then \enquote{$\left (f\circ g\right)(x)$ is defined whenever both $g(x)$ and $f(g(x))$ are defined.} Explain why both $g(x)$ and $f(g(x))$ must be defined in order to be able to write the function composition. Then find the formula for  $\left (f\circ g\right)(x)$, $\left (g\circ f\right)(x)$ and its domain, assuming that $f$ and $g$ are as in Problem \ref{item:one1}. 

\item Use Example 3 on pg.~213 to write $\displaystyle r(x)=(x-3)^2$ as the composition of two functions. What are the two functions, and in which order do you need to compose them? Redo the problem with $\displaystyle s(x)=x^2-3$.

\item If $\displaystyle f(x)=\frac{x}{x+1}$ and $\displaystyle g(x)=2x-1$, find the formulas for $f\circ g$, $g\circ f$, $f\circ f$ and $g\circ g$ and their domains (note: this is problem 55 on pg~217, which your textbook recommends to solve after reading Example 4 on pg. 214.)


\item What is a one-to-one function?

\item What is the horizontal line test? Can a function pass the horizontal line test but fail the vertical line test? Why? Why not? 

\item Suppose that $f$ is a one-to-one function: how does the inverse function of $f$ relate to $f$? What are the domain and range of the inverse function?

\item If $f$ is a function, is $\displaystyle f^{-1}(x)$ the same as $\displaystyle \frac{1}{f(x)}$? If your answer is yes, explain why. If it is no, give an example showing that the two functions are different.

\item Are all linear functions one-to-one? Why? Why not?

\item In Example 7 on pg.~222, your textbook shows how to find the inverse function of  $f(x)=3x-2$. Why does the book interchange $x$ and $y$?

\item In Week 2, we worked with the function described verbally as follows: \enquote{To evaluate F(x), subtract 1 from the input and divide the result by 2.}. Describe $\displaystyle F^{-1}(x)$ verbally. (Note: $\displaystyle F^{-1}(x)$ \enquote{undoes} $F(x)$.) 

\item This question is about the function $\displaystyle g(x)=x^2+2$. Please read and annotate Example~10 on pg.~224 before working on this problem.
\begin{enumerate}[labelindent=\parindent,leftmargin=*]
\item What are the domain and range of $g(x)$?
\item Enter {\tt{inverse function of g(x)=x\^{}2+2}} in \url{www.wolframalpha.com}. Does the graph given by WolframAlpha represent a function? Why? Why not?
\item In Example~10 on pg.~224, your book states that the inverse function for the right half of the parabola $\displaystyle y=x^2+2$ is the function $\displaystyle y=\sqrt{x-2}$. What is the inverse function for the left half of the parabola $\displaystyle y=x^2+2$?
\end{enumerate}

\item  At 9 am, water starts filling a swimming pool.
Two functions, $f$ and $g$, are described below.
\begin{description}
\item[$f(t)$:] The number of gallons of water that have been pumped  into the swimming pool $t$ minutes after 9 am.
\item[$g(x)$:] The depth of the water in the swimming pool, in inches, when it contains $x$ gallons of water.
\end{description}
Assume that $f$ and $g$ are one-to-one functions. 
\begin{enumerate}[labelindent=\parindent,leftmargin=*]
\item What does $\displaystyle g(f(20))$ represent in practical terms?
\item What does $\displaystyle f^{-1}(20)$ represent in practical terms?
\item What does $\displaystyle g^{-1}(35)$ represent in practical terms?  
\end{enumerate}
					
\item Two functions $f$ and $g$ are shown below.

\begin{minipage}{0.5\linewidth}
\begin{center}

\begin{mfpic}[20]{-3}{6}{-2}{5}

\polyline{(1,0), (5,2)} 
%\polyline{(2,0), (4,2)}
\point[5pt]{(1,0), (5,2)}
\tcaption{\scriptsize $y=f(x)$}
\axes
\xmarks{-2,-1,1,2,3,4,5}
\ymarks{-2,-1,1,2,3,4,}
\tlpointsep{4pt}
\axislabels {x}{{\tiny $-2$} -2,{\tiny $-1$} -1,{\tiny $1$} 1, {\tiny $2$} 2, {\tiny $3$} 3, {\tiny $4$} 4, {\tiny $5$} 5}
\axislabels {y}{{\tiny $1$} 1,{\tiny $2$} 2, {\tiny $3$} 3, {\tiny $4$} 4,  {\tiny $-1$} -1, {\tiny $-2$} -2}
\drawcolor[gray]{0.75} 
\grid{1,1}
\end{mfpic}
\end{center}
\end{minipage}
\begin{minipage}{0.5\linewidth}
\begin{center}

\begin{mfpic}[20]{-1}{6}{-2}{5}

%\polyline{(0,-2), (4,1), (4,2), (5,3)}

\polyline{(0,2), (4,0)} 

%\polyline{(3,3), (5,4)}

\point[5pt]{(0,2), (4,0)}
%\circle{(3, 3),0.15}
\tcaption{\scriptsize $y=g(x)$}
\axes

\xmarks{1,2,3,4,5}

\ymarks{-2,-1,1,2,3,4,}

\tlpointsep{4pt}

\axislabels {x}{{\tiny $1$} 1, {\tiny $2$} 2, {\tiny $3$} 3, {\tiny $4$} 4, {\tiny $5$} 5}

\axislabels {y}{{\tiny $1$} 1,{\tiny $2$} 2, {\tiny $3$} 3, {\tiny $4$} 4,  {\tiny $-1$} -1, {\tiny $-2$} -2}

\drawcolor[gray]{0.75} 
\grid{1,1}

\end{mfpic}
\end{center}
\end{minipage}

\begin{enumerate}
\item \label{ettob1}  Fill out the table:


\begin{minipage}{\linewidth}
\centering
\captionof{table}{$f(g(x))$} %\label{tab:title} 
\begin{tabularx}{0.8\textwidth}{|X|X|X|X|X|X|X|X|}
\hline
\multicolumn{2}{|c|}{$x$}         & $0$ & $1$ & $2$ & $3$ & $4$ & 5  \\ \hline
\multicolumn{2}{|c|}{$f(g(x))$}   & & &     &     &    &         \\ \hline
\end{tabularx}
\end{minipage}
\item What is the domain of $\displaystyle f(g(x))$?
\item Redo part \ref{ettob1} replacing $f(g(x))$ with $g(f(x))$.
\item What is the domain of $\displaystyle g(f(x))$?
\item Find formulas for $\displaystyle f^{-1}$ and $g^{-1}$. Find the domain and range of each function.
\end{enumerate}


\end{enumerate}
		



 \begin{mdframed}[style=exampledefault,frametitle={Looking Ahead to Next Week...}]
 	{\underline{Definitions that you should be familiar with by the next class meeting:} }
 	\begin{enumerate}[label= {  \arabic*:},labelindent=2em, style = standard,leftmargin=4pc, labelsep=*, noitemsep]
 		\item Composition of functions;
\item One-to-one function;
                     \item Inverse function.
 		
 		%\item Amplitude, period and midline.
 	\end{enumerate}
 	{\underline{You should be able to:} }
 	\begin{enumerate}[label= {  \arabic*:},labelindent=2em, style = standard,leftmargin=4pc, labelsep=*, noitemsep]
 		\item Find the domain of the sum, difference, product and quotient of two functions;
                     \item Write the formula for the composition of two functions;
\item Find the composition of functions using tables of functions and graphs;
                     \item Decide whether a function is one-to-one;
\item Find the inverse of a one-to one function; 
\item Find the domain and range of the inverse of a one-to-one function;
\item Explain the meaning of the inverse of a function in practical terms.
 	\end{enumerate}
{\underline{Advice on how to excel in precalculus:} }
\begin{enumerate}[label= {  \arabic*:},labelindent=2em, style = standard,leftmargin=4pc, labelsep=*, noitemsep]
 		\item  Always go through the discussion questions slowly again yourself after each class, to make sure you fully understand.
Check the  work you did on the questions before class against the answers provided in class to make sure you didn’t overlook anything.
\item Always read the sections we will cover{\bf{ before class}}.
\item Always attempt the discussion questionsr{\bf{ before class}}.
\item Give yourself sufficient time to do the homework: start early and don't just try to guess the answers. Work through each problem and take notes on how you attempt to solve the problem, why your attempt works or where you got stuck, and why.
\end{enumerate}
 	\end{mdframed}


\end{document} 
              